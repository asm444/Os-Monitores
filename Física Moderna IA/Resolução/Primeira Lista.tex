\documentclass[10pt,a4paper]{article}
\usepackage[utf8]{inputenc}
\usepackage[T1]{fontenc}
\usepackage{amsmath}
\usepackage{amssymb}
\usepackage{makeidx}
\usepackage{graphicx}
\usepackage[portuguese]{babel}
\author{Arthur de Souza Molina e Gabriel Capelini Magalhaes}
\title{Resolução da Lista e dos exercícios das notas.}
\begin{document}
	\maketitle
	\section{Primeiro Exercício}
	Qual a diferença entre os referenciais inerciais da Mecânica Clássica e os da Teoria da Relatividade Restrita?
	
	\paragraph{Resposta:}
	Na Mecânica Clássica, utilizamos as Transformações de Galileu como um dicionário para relacionar medidas entre referenciais inerciais. Na Teoria da Relatividade Restrita, utilizamos as Transformações de Lorentz para o mesmo propósito, uma vez que as TL satisfazem os postulados da TRR.
	
	%%%%%%%%%%%%%%%%%%%%%%%%%%%%
	
	\section{Terceiro Exercício}
	A distância até a estrela mais distante da nossa galáxia é da ordem de $10^{5}$ anos-luz. Explique por que é possível, em princípio, para um ser humano viajar para tal estrela durante seu tempo de vida (digamos 80 anos) e faça uma estimativa da velocidade necessária para isso.
	\paragraph{Resposta:}
	Antes de começar qualquer cálculo, perceba que as medidas de intervalo de tempo e de distância são de um observador em repouso no referencial da Terra. Este problema é análogo ao paradoxo dos gêmeos.
	
	Podemos definir o observador na Terra como o referencial S ( o observador na estrela está em repouso em relação ao referencial S) e o observador (o viajante) da nave em repouso em seu respectivo referencial S' que se distância da Terra com a velocidade iremos estimar, isto é, a velocidade relativa entre os referenciais S e S', lembrando que o sentido da velocidade do foguete se altera na mudança de referencial, logo,
	
	\begin{equation}\label{v_foguete}
		  v' = \dfrac{\Delta x'}{\Delta t'}
	\end{equation}
	
	Sabendo que o observador na Terra realiza medidas de comprimento contraídas e intervalos de tempo dilatadas
	\begin{equation}
		\Delta x \approx 10^5 \text{  anos-luz    e  } \Delta t \approx \text{80 anos},
	\end{equation}
	podemos usar as TL para traduzir as medidas para o sistema de coordenadas para S' com
	\begin{equation}\label{TL_repouso}
		\Delta x' = \dfrac{\Delta x}{\gamma}\text{  e  } \Delta t' = \gamma\Delta t.
	\end{equation}

	Ao substituirmos temos
	\begin{eqnarray}
		&& v' = \dfrac{\dfrac{\Delta x}{\gamma}}{\gamma\Delta t} \nonumber \\
		&& v' = \dfrac{\Delta x}{\gamma^2\Delta t}\nonumber
	\end{eqnarray}
	para facilitar as contas, podemos realizar a seguinte operação
	
	\begin{equation}\nonumber
		\dfrac{\Delta x}{\Delta t } = \dfrac{10^5 \text{  anos-luz}}{80\text{ anos}} = \dfrac{10^5c}{80} = 1,25\cdot10^3c
	\end{equation}
	irei definir $ \alpha\equiv 1,25\cdot10^3 $, apenas para não carregar um termo numérico imenso e dar o trabalho de manipula-lo. Retornando as contas temos
	
	\begin{eqnarray}
	    && v' = \dfrac{\alpha c}{\gamma^2}\nonumber \\
		&& v' = \alpha c \left( 1 - \dfrac{v'^2}{c^2}\right) \nonumber \\
		&& v'  =  \alpha c - \dfrac{ \alpha v'^2}{c} \nonumber \\		
		&& \dfrac{ c v'}{\alpha} = c^2 - v'^2 \nonumber \\
		&& v'^2 - \dfrac{ c v'}{\alpha} - c^2 = 0, \nonumber
	\end{eqnarray}
	usando Baskara, temos
	\begin{eqnarray}
		&& v' = -\dfrac{\frac{c}{\alpha} \pm \sqrt{\left(\frac{c}{\alpha}\right)^2 +4c^2}}{2} \nonumber \\
		&& v' = \dfrac{-\frac{c}{\alpha} \pm c\sqrt{\frac{1}{\alpha^2} +4}}{2} \nonumber 
	\end{eqnarray}
	como isso é uma estimativa, podemos considerar que $ \alpha^{-2} \approx 0 $, logo
	\begin{eqnarray}
		&& v' = \dfrac{-\frac{c}{\alpha} \pm c\sqrt{4}}{2} \nonumber \\
		&& v' = c\dfrac{-\frac{1}{\alpha} \pm 2}{2} \nonumber\\
		&& v' = c\left(-\frac{1}{2\alpha} \pm 1\right) \nonumber 
	\end{eqnarray}
	O primeiro resultado, o negativo, temos
	$ v' = c\left(-\frac{1}{2\alpha} - 1\right) \Longrightarrow |v'| > |c|$, ultrapassando a velocidade da luz, na direção contrária da viagem e isso é incorreto.
	
	Já o segundo resultado, o positivo, temos
	$ v' = c\left(-\frac{1}{2\alpha} + 1\right) \approx 0,9996 c$ que é o resultado correto.

	%%%%%%%%%%%%%%%%%%%%%%

	\section{Quinto Exercício}
	Uma barra de comprimento próprio $ L_0 $ (i.e., o comprimento medido no referencial $ S_0 $ em que ela está em repouso) faz um ângulo $ \theta_0 $ com o eixo
	horizontal de seu referencial. Para um observador em um referencial S que se desloca com relação à $ S_0 $ na direção horizontal com velocidade v:
	\begin{enumerate}
		\item Qual o valor do ângulo $ \theta $ que a barra faz com o eixo horizontal?
		
		\paragraph{Resposta:}
		Como $ S_0 $ se desloca com velocidade v na horizontal, isto é, no eixo x, portanto a componente vertical da barra ficará ilesa, pois a contração espacial ocorre apenas da direção do movimento.
		
		\begin{eqnarray}
			&& \theta_0 = \arctan\left(\frac{\Delta y_0}{\Delta x_0}\right) = \arctan\left(\frac{L_y}{L_x}\right)\nonumber \\
			&& L_0 = \sqrt{L_y^2+L_x^2} \nonumber\\
			&& \theta = \arctan\left(\frac{\Delta y}{\Delta x}\right) \nonumber \\
		\end{eqnarray}
		 Reescrevendo $ \frac{L_y}{L_x} \equiv \alpha$ em termos de $ \theta_0 $, temos
		 $$\theta_0 = \arctan\left(\frac{\Delta y_0}{\Delta x_0}\right) = \arctan\left(\frac{L_y}{L_x}\right) =  \arctan (\alpha)$$
		 
		 $$\therefore \alpha = \tan (\theta_0)$$
		 
		 Sabendo que $ \Delta y = \Delta y_0 = L_y $ e $\Delta x = \dfrac{\Delta x_0}{\gamma} =\dfrac{L_x}{\gamma} $, temos
		 
		 $$ \theta = \arctan\left(\frac{\Delta y}{\Delta x}\right) = \arctan\left(\frac{\Delta y_0}{ \dfrac{\Delta x_0}{\gamma}}\right) =\arctan\left(\frac{\Delta y_0}{ \dfrac{\Delta x_0}{\gamma}}\right) = \arctan(\gamma\alpha)$$
		 
		 $$\theta = \arctan(\gamma\alpha) = \arctan(\gamma\tan (\theta_0))$$
		 
		$$ \therefore \theta  = \arctan(\gamma\tan (\theta_0))$$
		
		\item Qual o comprimento L da barra em S?
		 \paragraph{Resposta:}
		 O comprimento da barra em $ S_0 $ é dado por
		 $$ L_0 = \sqrt{L_y^2+L_x^2}$$
		 e para o comprimento da barra em $ S $ é dado por
		 $$ L= \sqrt{L_y'^2+L_x'^2}$$
		 
		 $$ L= \sqrt{L_y^2+\left(\dfrac{L_x}{\gamma}\right)^2}$$
		 
		 $$ L= \sqrt{(L_x)^2\left(\left(\dfrac{L_y}{L_x}\right)^2+\left(\dfrac{1}{\gamma}\right)^2\right)} = L_x \sqrt{(\alpha)^2+\left(\dfrac{1}{\gamma}\right)^2} = L_x \sqrt{\tan^2 (\theta_0)+ 1 - 1 + \left(\dfrac{1}{\gamma}\right)^2} $$
		 
		 $$ L= L_x \sqrt{\sec^2 \theta_0 - 1 + \left(1 - \frac{v^2}{c^2}\right)} = L_x \sqrt{\sec^2 \theta_0 - \frac{v^2}{c^2}}$$
		 
		 podemos escrever $ L_x = L_0 \cos\theta_0$, ao substituir
		 
		 $$L=  L_x  \sqrt{\sec^2 \theta_0 - \frac{v^2}{c^2}} = L_0 \cos\theta_0\sqrt{\sec^2 \theta_0 - \frac{v^2}{c^2}} = L_0 \sqrt{\cos^2\theta_0\sec^2 \theta_0 - \frac{v^2\cos^2\theta_0}{c^2}}  $$
		 
		 $$\therefore L= L_0 \sqrt{1 - \frac{v^2\cos^2\theta_0}{c^2}}$$
	\end{enumerate}
	
	%%%%%%%%%%%%%%
	
	\section{Sétimo Exercício}
	Num certo referencial S um observador registra a seguinte sequência de eventos: num certo instante uma bomba explodiu numa certa posição e, 3 segundo depois, uma segunda bomba explodiu a uma distância de 1 metro
	desta primeira bomba. É possível encontrar um referencial inercial $ S_0 $,que respeite todos os postulados da relatividade, onde estas duas bombas explodiram no mesmo instante? Se sim, qual a velocidade deste referencial
	em relação a S. Se não, justifique sua resposta.
	
	\paragraph{Resposta:}
	Temos as medidas realizadas no referencial S
	
	 $$\Delta x = 1 m \,\,\,\, \Delta t = 3s $$
	 
	 precisamos encontrar um referencial S' em que um observador registra as duas explosões ocorrendo simultaneamente ($ \Delta t' = 0 $), ao usar uma das transformações de Lorentz podemos encontrar a velocidade relativa entre os referenciais S e S'.
	 
	 $$ \Delta t' = \gamma \left( \Delta t - v \dfrac{\Delta x}{c^2}\right)$$

	 $$ 0 = \gamma \left( 3 - v \dfrac{1}{c^2}\right) $$
	 
	 $$ 0 =  3 - v \dfrac{1}{c^2} $$
	 
	 $$ v \dfrac{1}{c^2} = 3$$
	 
	 $$ v = 3c^2$$
	O referencial S' precisaria estar a uma velocidade relativa ao referencial S de $ 3c^2 $, portanto não é encontrar um referencial que consiga registrar os dois eventos ocorrendo simultaneamente.
	
	Outra maneira de resolver esse problema é verificar qual tipo de distância que temos entre os eventos
	
	$$( (\Delta s)^2 = \Delta x)^2 - c^2 (\Delta t)^2 = 1^2 -9c^2 < 0 $$
	portanto, a distância entre os evento é do tipo-tempo, ou seja, há uma relação de causalidade entre os eventos, logo é impossível encontrar um referencial que consiga ver os dois eventos ocorrerem simultaneamente.
	

	
\end{document}