 	\documentclass[10pt,a4paper]{article}
\usepackage[utf8]{inputenc}
\usepackage[T1]{fontenc}
\usepackage{amsmath}
\usepackage{amssymb}
\usepackage{makeidx}
\usepackage{graphicx}
\usepackage{siunitx}
\usepackage{mathtools}
\usepackage[portuguese]{babel}
\author{Arthur de Souza Molina e Gabriel Capelini Magalhaes}
\title{Resolução da Lista e dos exercícios das notas.}
\begin{document}
	\maketitle
	\section{Primeiro Exercício}
	Qual a diferença entre os referenciais inerciais da Mecânica Clássica e os da Teoria da Relatividade Restrita?
	
	\paragraph{Resposta:}
	Na Mecânica Clássica, utilizamos as Transformações de Galileu como um dicionário para relacionar medidas entre referenciais inerciais. Na Teoria da Relatividade Restrita, utilizamos as Transformações de Lorentz para o mesmo propósito, uma vez que as TL satisfazem os postulados da TRR.
	
	%%%%%%%%%%%%%%%%%%%%%%%%%%%%
	
	\section{Segundo Exercício}
	Até aqui nós escrevemos apenas transformações de Lorentz na direção \(x\).
	Como seria uma transformação de Lorentz na direção \(y\)?	
	\paragraph{Resposta:}
	A transformação de Lorentz para a direção \(y\) tem a mesma forma algébrica que para a direção \(x\), porém agora como o movimento se da apenas em \(y\), então as TL ficam como
	
	\begin{equation}
		\begin{split}
			& y' = \gamma(y-vt) \\
			& x'=x, \;\; z'=z \\
			& t' = \gamma\left(t-\frac{v}{c^2}y'\right)
		\end{split}
	\end{equation}
%%%%%%%%%%%%%%%%%%%%%%%%%
	
	\section{Terceiro Exercício}
	A distância até a estrela mais distante da nossa galáxia é da ordem de $10^{5}$ anos-luz. Explique por que é possível, em princípio, para um ser humano viajar para tal estrela durante seu tempo de vida (digamos 80 anos) e faça uma estimativa da velocidade necessária para isso.
	\paragraph{Resposta:}
	Antes de começar qualquer cálculo, perceba que as medidas de intervalo de tempo e de distância são de um observador em repouso no referencial da Terra.
	
	Podemos definir o observador na Terra como o referencial S ( o observador na estrela está em repouso em relação ao referencial S) e o observador (o viajante) da nave em repouso em seu respectivo referencial S' que se distância da Terra com a velocidade que iremos estimar, isto é, a velocidade relativa entre os referenciais S e S'. Lembrando que o sentido da velocidade do foguete se altera na mudança de referencial, logo,
	
	\begin{equation}\label{v_foguete}
		  v' = \dfrac{\Delta x'}{\Delta t'}
	\end{equation}
	
	Sabendo que o observador na Terra realiza medidas de comprimento contraídas e intervalos de tempo dilatados
	\begin{equation}
		\Delta x \approx 10^5 \text{  anos-luz    e  } \Delta t \approx \text{80 anos},
	\end{equation}
	podemos usar as TL para traduzir as medidas para o sistema de coordenadas para S' com
	\begin{equation}\label{TL_repouso}
		\Delta x' = \dfrac{\Delta x}{\gamma}\text{  e  } \Delta t' = \gamma\Delta t.
	\end{equation}

	Ao substituirmos temos
	\begin{eqnarray}
		&& v' = \dfrac{\dfrac{\Delta x}{\gamma}}{\gamma\Delta t} \nonumber \\
		&& v' = \dfrac{\Delta x}{\gamma^2\Delta t}\nonumber
	\end{eqnarray}
	para facilitar as contas, podemos realizar a seguinte operação
	
	\begin{equation}\nonumber
		\dfrac{\Delta x}{\Delta t } = \dfrac{10^5 \text{  anos-luz}}{80\text{ anos}} = \dfrac{10^5c}{80} = 1,25\cdot10^3c
	\end{equation}
	irei definir $ \alpha\equiv 1,25\cdot10^3 $, apenas para não carregar um termo numérico imenso e dar o trabalho de manipula-lo. Retornando as contas temos
	
	\begin{eqnarray}
	    && v' = \dfrac{\alpha c}{\gamma^2}\nonumber \\
		&& v' = \alpha c \left( 1 - \dfrac{v'^2}{c^2}\right) \nonumber \\
		&& v'  =  \alpha c - \dfrac{ \alpha v'^2}{c} \nonumber \\		
		&& \dfrac{ c v'}{\alpha} = c^2 - v'^2 \nonumber \\
		&& v'^2 - \dfrac{ c v'}{\alpha} - c^2 = 0, \nonumber
	\end{eqnarray}
	usando Bhaskara, temos
	\begin{eqnarray}
		&& v' = -\dfrac{\frac{c}{\alpha} \pm \sqrt{\left(\frac{c}{\alpha}\right)^2 +4c^2}}{2} \nonumber \\
		&& v' = \dfrac{-\frac{c}{\alpha} \pm c\sqrt{\frac{1}{\alpha^2} +4}}{2} \nonumber 
	\end{eqnarray}
	como isso é uma estimativa, podemos considerar que $ \alpha^{-2} \approx 0 $, logo
	\begin{eqnarray}
		&& v' = \dfrac{-\frac{c}{\alpha} \pm c\sqrt{4}}{2} \nonumber \\
		&& v' = c\dfrac{-\frac{1}{\alpha} \pm 2}{2} \nonumber\\
		&& v' = c\left(-\frac{1}{2\alpha} \pm 1\right) \nonumber 
	\end{eqnarray}
	O primeiro resultado, o negativo, temos
	$ v' = c\left(-\frac{1}{2\alpha} - 1\right) \Longrightarrow |v'| > |c|$, ultrapassando a velocidade da luz, na direção contrária da viagem e isso é incorreto.
	
	Já o segundo resultado, o positivo, temos
	$ v' = c\left(-\frac{1}{2\alpha} + 1\right) \approx 0,9996 c$ que é o resultado correto.

	%%%%%%%%%%%%%%%%%%%%%%
	
	\section{Quarto Exercício}	
	Considere o problema do muon descrito nas notas de aula. Lembrando que o muon tem um tempo médio de decaimento de $2,2\times 10^{-6}$s. Com que
	velocidade ele precisa ser produzido na atmosfera para percorrer 10 km e atingir o nível do mar?
	
	\paragraph{Resposta:}
	Para esse problema, precisamos compreender que o tempo de  decaimento do múon é o tempo que ele vai percorrer \textbf{no seu referencial}. Vamos considerar então o referencial do múon como \(S\) e o referencial da Terra com \(S'\). Obviamente \(S'\) se move com relação ao múon, e em decorrência disso, a distância percorrida pelo múon será 
	
	\begin{equation*}
		\Delta x = \frac{1}{\gamma}\Delta x'
	\end{equation*}
	
	
	De modo que a sua velocidade será 
	
	\begin{equation*}
		v = \frac{\Delta x}{\Delta t} = \frac{1}{\gamma}\frac{\Delta x'}{\Delta t} = \sqrt{1-\frac{v^2}{c^2}}\frac{\Delta x'}{\Delta t} 
	\end{equation*}
	
	definindo $\alpha = \dfrac{\Delta x'}{\Delta t}$ por praticidade e resolvendo para \(v\)
	
	\begin{equation*}
		\begin{split}
			&v^2 = \left(1 -\frac{v^2}{c^2}\right)\alpha^2 \Rightarrow v^2 + \frac{v^2}{c^2}\alpha^2 = \alpha^2 \Rightarrow v^2\left[1 + \left(\frac{\alpha}{c}\right)^2\right] = \alpha^2 \Rightarrow v = \frac{\alpha}{\sqrt{1+\frac{\alpha^2}{c^2}}}
		\end{split}
	\end{equation*}
	
	onde 
	
	\begin{equation}\nonumber
		\alpha = \frac{\Delta x'}{\Delta t} = \frac{10^4 \si{m}}{2,2\times 10^{-6}\si{s}} = 4,54\times 10^9\si{m/s} \quad \text{e} \quad c\approx 3\times 10^8\si{m/s}
	\end{equation}
	
	então
	
	\begin{equation}\nonumber
		v = \frac{4,54\times 10^9\si{m/s}}{\sqrt{1+\left(\frac{4,54\times 10^9\si{m/s}}{3\times 10^8}\right)^2}} \approx 2,9934 \times 10^8 \si{m/s} 
	\end{equation}
%%%%%%%%%%%%%%%%%%%%%%%%%%%%%

	\section{Quinto Exercício}
	Uma barra de comprimento próprio $ L_0 $ (i.e., o comprimento medido no referencial $ S_0 $ em que ela está em repouso) faz um ângulo $ \theta_0 $ com o eixo
	horizontal de seu referencial. Para um observador em um referencial S que se desloca com relação à $ S_0 $ na direção horizontal com velocidade v:
	\begin{enumerate}
		\item[(a)] Qual o valor do ângulo $ \theta $ que a barra faz com o eixo horizontal?
		
		\paragraph{Resposta:}
		Como $ S_0 $ se desloca com velocidade v na horizontal, isto é, no eixo x, portanto a componente vertical da barra ficará ilesa, pois a contração espacial ocorre apenas da direção do movimento.
		
		\begin{eqnarray}
			&& \theta_0 = \arctan\left(\frac{\Delta y_0}{\Delta x_0}\right) = \arctan\left(\frac{L_y}{L_x}\right)\nonumber \\
			&& L_0 = \sqrt{L_y^2+L_x^2} \nonumber\\
			&& \theta = \arctan\left(\frac{\Delta y}{\Delta x}\right) \nonumber \\
		\end{eqnarray}
		 Reescrevendo $ \frac{L_y}{L_x} \equiv \alpha$ em termos de $ \theta_0 $, temos
		 $$\theta_0 = \arctan\left(\frac{\Delta y_0}{\Delta x_0}\right) = \arctan\left(\frac{L_y}{L_x}\right) =  \arctan (\alpha)$$
		 
		 $$\therefore \alpha = \tan (\theta_0)$$
		 
		 Sabendo que $ \Delta y = \Delta y_0 = L_y $ e $\Delta x = \dfrac{\Delta x_0}{\gamma} =\dfrac{L_x}{\gamma} $, temos
		 
		 $$ \theta = \arctan\left(\frac{\Delta y}{\Delta x}\right) = \arctan\left(\frac{\Delta y_0}{ \dfrac{\Delta x_0}{\gamma}}\right) = \arctan(\gamma\alpha)$$
		 
		 $$\theta = \arctan(\gamma\alpha) = \arctan(\gamma\tan (\theta_0))$$
		 
		$$ \therefore \theta  = \arctan(\gamma\tan (\theta_0))$$
		
		\item[(b)] Qual o comprimento L da barra em S?
		 \paragraph{Resposta:}
		 O comprimento da barra em $ S_0 $ é dado por
		 $$ L_0 = \sqrt{L_y^2+L_x^2}$$
		 e para o comprimento da barra em $ S $ é dado por
		 $$ L= \sqrt{L_y'^2+L_x'^2}$$
		 
		 $$ L= \sqrt{L_y^2+\left(\dfrac{L_x}{\gamma}\right)^2}$$
		 
		 $$ L= \sqrt{(L_x)^2\left(\left(\dfrac{L_y}{L_x}\right)^2+\left(\dfrac{1}{\gamma}\right)^2\right)} = L_x \sqrt{(\alpha)^2+\left(\dfrac{1}{\gamma}\right)^2} = L_x \sqrt{\tan^2 (\theta_0)+ 1 - 1 + \left(\dfrac{1}{\gamma}\right)^2} $$
		 
		 $$ L= L_x \sqrt{\sec^2 \theta_0 - 1 + \left(1 - \frac{v^2}{c^2}\right)} = L_x \sqrt{\sec^2 \theta_0 - \frac{v^2}{c^2}}$$
		 
		 podemos escrever $ L_x = L_0 \cos\theta_0$, ao substituir
		 
		 $$L=  L_x  \sqrt{\sec^2 \theta_0 - \frac{v^2}{c^2}} = L_0 \cos\theta_0\sqrt{\sec^2 \theta_0 - \frac{v^2}{c^2}} = L_0 \sqrt{\cos^2\theta_0\sec^2 \theta_0 - \frac{v^2\cos^2\theta_0}{c^2}}  $$
		 
		 $$\therefore L= L_0 \sqrt{1 - \frac{v^2\cos^2\theta_0}{c^2}}$$
	\end{enumerate}
	
	%%%%%%%%%%%%%%
	
	\section{Sexto Exercício}
	Se, para o observador em $S$, um feixe luminoso viaja por um intervalo de
	tempo $\Delta t$, esse feixe percorre uma distância $\Delta x = c\Delta t$. Da mesma forma,
	para um observador em $S'$ esse \textbf{mesmo feixe} viaja por um intervalo de
	tempo $\Delta t'$, percorrendo uma distância $\Delta x'$. Se a velocidade da luz é a mesma para ambos os observadores temos 
	
	\begin{equation*}
		\begin{rcases}
			\Delta x = c\Delta t \Rightarrow \dfrac{\Delta x}{\Delta t} = c \\
			\Delta x' = c\Delta t' \Rightarrow \dfrac{\Delta x'}{\Delta t'} = c
		\end{rcases} 
		\frac{\Delta x}{\Delta t} = \frac{\Delta x'}{\Delta t'}
	\end{equation*}
	
	Se usarmos agora a dilatação temporal para um dos observadores (qualquer um), e.g.,$\Delta t =\gamma \Delta t'$ temos 
	
	\begin{equation}\label{dilatacao espacial}
		\Delta t = \gamma \Delta t' \Rightarrow \frac{\Delta x}{\gamma \Delta t'}=\frac{\Delta x'}{\Delta t'} \Rightarrow \Delta x = \gamma \Delta x'
	\end{equation}
	
	Vimos que, devido aos efeitos relativísticos, devemos esperar uma dilatação temporal $(\Delta t = \gamma \Delta t')$ e uma contração espacial $(\Delta x = \Delta x'/\gamma)$. Mas a expressão acima mostra que, se o tempo se dilata $\Delta t = \gamma \Delta t'$, o espaço também deve se dilatar $\Delta x = \gamma \Delta x'$. Responda, argumentando
	com detalhes, as seguintes questões:
	
	\begin{itemize}
		\item[(a)] Se a velocidade da luz é constante o espaço deve se contrair, ou se dilatar?
		
		\paragraph{Resposta:} Por conta da constância da velocidade da luz, o espaço deve se contrair, pois \textbf{uma medida de distância deve ser feita simultaneamente}, ou seja, os dois pontos devem ser medidos no mesmo instante. Mas como sabemos, para um referencial que se move com relação ao que faz as medidas simultaneamente, certamente ele não conseguirá fazer medidas ao mesmo tempo. O resultado disso é uma medida contraída para este.
		
		
		\item[(b)] Como você explica (i.e qual o significado Físico) da dilatação espacial em (\ref{dilatacao espacial})?
		
		\paragraph{Resposta:}  Perceba que quando usamos a dilatação do tempo ($\Delta t = \gamma \Delta t'$), precisamos entender para qual observador usamos. Nesse caso, como $\gamma \Delta t' > \Delta t'$, \textbf{estamos fazendo as descrições do observador no referencial $\mathbf{S'}$}, pois para ele, o tempo de $S$ passa mais devagar. Mas veja que em (\ref{dilatacao espacial}) estamos dizendo \textbf{como o observador em $\mathbf{S}$ vê os comprimentos de $\mathbf{S'}$}. Porém essa interpretação está errada já que estamos fazendo a dilatação para observador de $S'$. Dito isso, se dividirmos (\ref{dilatacao espacial}) por $\gamma$, obtemos 
		
		\begin{equation}
			\Delta x' = \frac{1}{\gamma}\Delta x
		\end{equation}
		
		O que agora entra de acordo com o que esperamos.
		
		\textbf{Obs:} Obviamente podemos usar a dilatação espacial para qualquer um dos observadores. Logo, se usarmos a inversa ($\Delta t' = \gamma \Delta t$), devemos verificar que $\Delta x = \frac{1}{\gamma}\Delta x'$.
	\end{itemize}
	
	%%%%%%%%%%%%%%
	
	\section{Sétimo Exercício}
	Num certo referencial S um observador registra a seguinte sequência de eventos: num certo instante uma bomba explodiu numa certa posição e, 3 segundo depois, uma segunda bomba explodiu a uma distância de 1 metro
	desta primeira bomba. É possível encontrar um referencial inercial $ S_0 $,que respeite todos os postulados da relatividade, onde estas duas bombas explodiram no mesmo instante? Se sim, qual a velocidade deste referencial
	em relação a S. Se não, justifique sua resposta.
	
	\paragraph{Resposta:}
	Temos as medidas realizadas no referencial S
	
	 $$\Delta x = 1 m \,\,\,\, \Delta t = 3s $$
	 
	 precisamos encontrar um referencial S' em que um observador registra as duas explosões ocorrendo simultaneamente ($ \Delta t' = 0 $), ao usar uma das transformações de Lorentz podemos encontrar a velocidade relativa entre os referenciais S e S'.
	 
	 $$ \Delta t' = \gamma \left( \Delta t - v \dfrac{\Delta x}{c^2}\right)$$

	 $$ 0 = \gamma \left( 3 - v \dfrac{1}{c^2}\right) $$
	 
	 $$ 0 =  3 - v \dfrac{1}{c^2} $$
	 
	 $$ v \dfrac{1}{c^2} = 3$$
	 
	 $$ v = 3c^2$$
	O referencial S' precisaria estar a uma velocidade relativa ao referencial S de $ 3c^2 $, portanto não é possível encontrar um referencial que consiga registrar os dois eventos ocorrendo simultaneamente.
	
	Outra maneira de resolver esse problema é verificar qual tipo de intervalo que temos entre os eventos
	
	$$( (\Delta s)^2 = \Delta x)^2 - c^2 (\Delta t)^2 = 1^2 -9c^2 < 0 $$
	portanto, a distância entre os evento é do tipo-tempo, ou seja, há uma relação de causalidade entre os eventos, logo é impossível encontrar um referencial que consiga ver os dois eventos ocorrerem simultaneamente.
	
	%%%%%%%%%%%%%%%%%%%%%%
	
	\section{Oitavo Exercício}
	Considere a seguinte transformação entre os referenciais: 
	\begin{align*}
	&x' = \frac{1}{2}\left[\gamma(x-z)+x+z-\gamma tv\sqrt{2}\right] \\
	&z'=\frac{1}{2}\left[\gamma(z-x)+x+z+\gamma tv\sqrt{2}\right]\\
	&t'=\frac{\gamma}{\sqrt{2}}\left[\frac{v}{c^2}(z-x)+\sqrt{2}t\right] \\
	&y'=y, \;\; \gamma = (1-\beta^2)^{-1/2}, \;\; \beta = \frac{v}{c}
	\end{align*}
	
	Essa transformação é compatível com o segundo postulado de Einstein? 
	
	\paragraph{Resposta:} Substituir as transformações dadas no invariante $ds^2$ e verificar que 
	
	\begin{equation*}
	 ds^2 = ds'^2
	\end{equation*}
	%%%%%%%%%%%%%%%%%%%%%
	
	\section{Nono Exercício}
	Num sistema onde nomeamos as coordenadas como
	$$x_0 = ct,\,\,x_1 =x,\,\,x_2=y,\,\,x_3=z$$
	considere a seguinte transformação linear
	$$x_\mu = \sum_{\nu=0}^{3} M_{\mu\nu} x_\nu,\,\,\,\mu =0,1,2,3.$$
	\begin{enumerate}
		\item[(a)] Determine as condições que $ M_{\mu\nu} $ deve respeitar para que
		
		$$ \text{\~{x}}^2_0 - \sum_{i=1}^{3} \text{\~{x}}^2_i = \text{x}^2_0 - \sum_{i=1}^{3} \text{x}^2_i $$
	
	\paragraph{Resposta:}
	Abrindo as componentes da transformação linear temos
	
	$$ x_\mu = M_{\mu0} x_0 + M_{\mu1} x_1+M_{\mu2} x_2+M_{\mu3} x_3 $$
	para $ \mu =0 $
	$$ \text{\~{x}}_0 = M_{00} x_0 + M_{01} x_1+M_{02} x_2+M_{03} x_3 $$
	para $ \mu =1 $
	$$ \text{\~{x}}_1 = M_{10} x_0 + M_{11} x_1+M_{12} x_2+M_{13} x_3 $$
	para $ \mu =2 $
	$$\text{\~{x}}_2 = M_{20} x_0 + M_{21} x_1+M_{22} x_2+M_{23} x_3 $$
	para $ \mu =3 $
	$$ \text{\~{x}}_3 = M_{30} x_0 + M_{31} x_1+M_{32} x_2+M_{33} x_3 $$
	 
	 calculando o $( \text{\~{x}}_\mu)^2 $ para substituir na equação acima.
	 
	 Para cada $ \mu=0 $
	 $$ (\text{\~{x}}_0)^2 =  (M_{00} x_0 + M_{01} x_1+M_{02} x_2+M_{03} x_3)^2 $$
	 \begin{equation}\nonumber
	 	\begin{split}
	 		(\text{\~{x}}_0)^2 =  M_{00}^2 x_0^2 + M_{00}M_{01} x_0x_1+M_{02}M_{00} x_0 x_2+M_{03}M_{00} x_0 x_3\\ 
	 		M_{00}M_{01} x_1 x_0 + M_{01}^2 x_1^2+M_{02}M_{01} x_1 x_2+M_{03}M_{01} x_1 x_3\\
	 		M_{00}M_{02} x_2 x_0 + M_{01}M_{02} x_2 x_1+M_{02}^2x_2^2+M_{03} M_{02} x_2x_3\\
	 		M_{00}M_{03} x_3 x_0 + M_{01}M_{03} x_3 x_1+M_{02}M_{03} x_3 x_2+M_{03}^2 x_3^2
	 	\end{split}
	 \end{equation}
 
	 \begin{equation}\nonumber
	 	\begin{split}
	 		(\text{\~{x}}_0)^2 =  M_{00}^2 x_0^2 +M_{01}^2 x_1^2+M_{02}^2x_2^2+M_{03}^2 x_3^2+ 2M_{00}M_{01} x_0x_1+2M_{02}M_{00} x_0 x_2\\+2M_{01}M_{03} x_3 x_1+2M_{03}M_{00} x_0 x_3+2M_{02}M_{01} x_1 x_2+2M_{02}M_{03} x_3 x_2
	 	\end{split}
	 \end{equation}
 
 	para $ \mu =1 $
 	$$ (\text{\~{x}}_1)^2 = (M_{10} x_0 + M_{11} x_1+M_{12} x_2+M_{13} x_3)^2 $$
 	\begin{equation}\nonumber
 		\begin{split}
 			(\text{\~{x}}_1)^2 = M_{10}^2 x_0^2 + M_{11}M_{10} x_0 x_1+M_{12} M_{10} x_0x_2+M_{13}M_{10} x_0 x_3\\
 			M_{10}M_{11} x_1 x_0 + M_{11}^2 x_1^2+M_{12}M_{11} x_1 x_2+M_{13}M_{11} x_1 x_3\\
 			M_{10}M_{12} x_2 x_0 + M_{11}M_{12} x_2 x_1+M_{12}^2 x_2^2+M_{13}M_{12} x_2 x_3\\
 			M_{10}M_{13} x_3x_0 + M_{11}M_{13} x_3 x_1+M_{12} M_{13} x_3x_2+M_{13}^2 x_3^2
 		\end{split}
 	\end{equation}
 
 	\begin{equation}\nonumber
 		\begin{split}
 			(\text{\~{x}}_1)^2 = M_{10}^2 x_0^2 + M_{11}^2+M_{12}^2 x_2^2+M_{13}^2 x_3^2+2M_{11}M_{10} x_0 x_1+2M_{12} M_{10} x_0x_2\\+2M_{13}M_{10} x_0 x_3+2M_{11}M_{13} x_3 x_1+2M_{12} M_{13} x_3x_2+2M_{12}M_{11} x_1 x_2
 		\end{split}
 	\end{equation}
	
	para $ \mu =2 $
	$$( \text{\~{x}}_2)^2 = (M_{20} x_0 + M_{21} x_1+M_{22} x_2+M_{23} x_3)^2 $$
	
	\begin{equation}\nonumber
		\begin{split}
			( \text{\~{x}}_2)^2 = M_{20}^2 x_0^2 + M_{21}M_{20} x_0 x_1+M_{22} M_{20} x_0x_2+M_{23} M_{20} x_0x_3\\
			M_{20}M_{21} x_1 x_0 + M_{21}^2 x_1^2+M_{22} M_{21} x_1x_2+M_{23}M_{21} x_1 x_3\\
			M_{20} x_0 M_{22} x_2+ M_{21} M_{22} x_2x_1+M_{22}^2 x_2^2+M_{23} M_{22} x_2x_3\\
			M_{20}M_{23} x_3 x_0 + M_{21} M_{23} x_3x_1+M_{22}M_{23} x_3 x_2+M_{23}^2 x_3^2	
		\end{split}
	\end{equation}

\begin{equation}\nonumber
	\begin{split}
		( \text{\~{x}}_2)^2 = M_{20}^2 x_0^2 + M_{21}^2 x_1^2+M_{22}^2 x_2^2+M_{23}^2 x_3^2	+2 M_{21}M_{20} x_0 x_1+2M_{22} M_{20}
		 x_0x_2\\+2M_{23} M_{20} x_0x_3+2M_{21} M_{23} x_3x_1+2M_{22}M_{23} x_3 x_2+2 M_{21} M_{22} x_2x_1
	\end{split}
\end{equation}

	para $ \mu =3 $
	$$ (\text{\~{x}}_3)^2 =( M_{30} x_0 + M_{31} x_1+M_{32} x_2+M_{33} x_3)^2 $$
	
	\begin{equation}\nonumber
		\begin{split}
			(\text{\~{x}}_3)^2 =M_{30}^2 x_0^2 + M_{31}M_{30} x_0 x_1+M_{32} M_{30} x_0x_2+M_{33}M_{30} x_0 x_3\\
			M_{30}M_{31} x_1 x_0 + M_{31}^2 x_1^2+M_{32}M_{31} x_1 x_2+M_{33}M_{31} x_1 x_3\\
			M_{30}M_{32} x_2 x_0 + M_{31} M_{32} x_2x_1+M_{32}^2 x_2^2+M_{33}M_{32} x_2 x_3\\
			 M_{30}M_{33} x_3 x_0 + M_{31} M_{33} x_3x_1+M_{32}M_{33} x_3 x_2+M_{33}^2 x_3^2
		\end{split}
	\end{equation}

		\begin{equation}\nonumber
		\begin{split}
			(\text{\~{x}}_3)^2 =M_{30}^2 x_0^2 + M_{31}^2 x_1^2+M_{32}^2 x_2^2+M_{33}^2 x_3^2+2M_{31}M_{30} x_0 x_1+2M_{32} M_{30} x_0x_2\\+2M_{33}M_{30} x_0 x_3+2 M_{31} M_{33} x_3x_1+2M_{32}M_{33} x_3 x_2+2M_{33}M_{32} x_2 x_3
		\end{split}
	\end{equation}
	
	subsituindo, temos
	\begin{equation}\nonumber
		\begin{split}
			M_{00}^2 x_0^2 +M_{01}^2 x_1^2+M_{02}^2x_2^2+M_{03}^2 x_3^2+ 2M_{00}M_{01} x_0x_1+2M_{02}M_{00} x_0 x_2\\+2M_{01}M_{03} x_3 x_1+2M_{03}M_{00} x_0 x_3+2M_{02}M_{01} x_1 x_2+2M_{02}M_{03} x_3 x_2 \\
			-(M_{10}^2 x_0^2 + M_{11}^2+M_{12}^2 x_2^2+M_{13}^2 x_3^2+2M_{11}M_{10} x_0 x_1+2M_{12} M_{10} x_0x_2\\+2M_{13}M_{10} x_0 x_3+2M_{11}M_{13} x_3 x_1+2M_{12} M_{13} x_3x_2+2M_{12}M_{11} x_1 x_2\\
			M_{20}^2 x_0^2 + M_{21}^2 x_1^2+M_{22}^2 x_2^2+M_{23}^2 x_3^2	+2 M_{21}M_{20} x_0 x_1+2M_{22} M_{20}
			x_0x_2\\+2M_{23} M_{20} x_0x_3+2M_{21} M_{23} x_3x_1+2M_{22}M_{23} x_3 x_2+2 M_{21} M_{22} x_2x_1\\
			M_{30}^2 x_0^2 + M_{31}^2 x_1^2+M_{32}^2 x_2^2+M_{33}^2 x_3^2+2M_{31}M_{30} x_0 x_1+2M_{32} M_{30} x_0x_2\\+2M_{33}M_{30} x_0 x_3+2 M_{31} M_{33} x_3x_1+2M_{32}M_{33} x_3 x_2+2M_{33}M_{32} x_2 x_3) \\
			= \text{x}^2_0 -  \text{x}^2_1 -  \text{x}^2_2 -  \text{x}^2_3
		\end{split}
	\end{equation}
	Agrupando os termos e relacionando com os termos do lado direito da equação, obtemos
	
	\begin{eqnarray}
		&& (M_{00}^2-M_{10}^2-M_{20}^2-M_{30}^2)x^2_0 = x^2_0\nonumber \\
		&& (M_{01}^2-M_{11}^2-M_{21}^2-M_{31}^2)x^2_1 = x^2_1\nonumber \\
		&& (M_{02}^2-M_{12}^2-M_{22}^2-M_{32}^2)x^2_2 = x^2_2\nonumber \\
		&& (M_{03}^2-M_{13}^2-M_{23}^2-M_{33}^2)x^2_3 = x^2_3\nonumber \\
		&& 2(M_{00}M_{01}-M_{11}M_{10}-M_{21}M_{20}-M_{31}M_{30})x_0x_1 = 0 x_0x_1\nonumber \\
		&&2(M_{00}M_{02}- M_{12}M_{10}-M_{22}M_{20}-M_{32}M_{30})x_0x_2 = 0 x_0x_2\nonumber \\
		&&2(M_{00}M_{03}- M_{13}M_{10}-M_{23}M_{20}-M_{33}M_{30})x_0x_3 = 0 x_0x_3\nonumber \\
		&&2(M_{01}M_{02}- M_{11}M_{12}-M_{21}M_{22}- M_{31}M_{32})x_1x_2 = 0x_1x_2\nonumber \\
		&& 2(M_{01}M_{03}-M_{11}M_{13}-M_{21}M_{23}-M_{31}M_{33})x_1x_3 = 0 x_1x_3\nonumber \\
		&& 2(M_{02}M_{03}-M_{12}M_{13}-M_{22}M_{23}-M_{32}M_{33})x_2x_3 = 0 x_2x_3 \nonumber
	\end{eqnarray}

	As condições que $ M_{\mu\nu} $ deve respeitar são
	\begin{eqnarray}
		&& M_{00}^2-M_{10}^2-M_{20}^2-M_{30}^2 = 1\nonumber \\
		&&M_{01}^2-M_{11}^2-M_{21}^2-M_{31}^2 = 1\nonumber \\
		&&M_{02}^2-M_{12}^2-M_{22}^2-M_{32}^2 = 1 \nonumber \\
		&& M_{03}^2-M_{13}^2-M_{23}^2-M_{33}^2 = 1\nonumber \\
		&& M_{00}M_{01}-M_{11}M_{10}-M_{21}M_{20}-M_{31}M_{30} = 0 \nonumber \\
		&&M_{00}M_{02}- M_{12}M_{10}-M_{22}M_{20}-M_{32}M_{30}= 0 \nonumber \\
		&&M_{00}M_{03}- M_{13}M_{10}-M_{23}M_{20}-M_{33}M_{30} = 0 \nonumber \\
		&&M_{01}M_{02}- M_{11}M_{12}-M_{21}M_{22}- M_{31}M_{32} = 0\nonumber \\
		&& M_{01}M_{03}-M_{11}M_{13}-M_{21}M_{23}-M_{31}M_{33}= 0 \nonumber \\
		&& M_{02}M_{03}-M_{12}M_{13}-M_{22}M_{23}-M_{32}M_{33} = 0  \nonumber
	\end{eqnarray}
	
	\item[(b)] Agora escreva
	\begin{equation}\nonumber
		M =\left(\begin{matrix}
			a && \mathbf{b}^{T} \\
			\mathbf{c} && D
		\end{matrix}\right)
	\end{equation}
com $ a $ um número $ D $ uma matriz $ 3 \times 3 $ e
\begin{equation}\nonumber
	\mathbf{b} = \left(\begin{matrix}
		b1 \\
		b2 \\
		b3
	\end{matrix}\right) \,\, , \,\, 	\mathbf{c} = \left(\begin{matrix}
	c1 \\
	c2 \\
	c3
\end{matrix}\right).
\end{equation}
Mostre que as condições obtidas no item anterior podem ser escritas como
\begin{equation}\nonumber
	M^T\eta M = B \text{   onde } \eta = \left( \begin{matrix}
		1 && 0 \\
		0 && -I
	\end{matrix}\right).
\end{equation}
\paragraph{Resposta:}
Calculando o produto matricial $ \eta M $, obtemos

\begin{equation}\nonumber
	M = \left(\begin{matrix}
		a && b_1 && b_2 && b_3 \\
		c_1 && D_{11} && D_{12} && D_{13} \\
		c_2 && D_{21} && D_{22} && D_{23}\\
		c_3 && D_{31} && D_{32} && D_{33}
	\end{matrix}\right) 
\end{equation}

\begin{equation}\nonumber
	\left(\begin{matrix}
		1 && 0 && 0 && 0 \\
		0 && -1 && 0 && 0 \\
		0 && 0 && -1 && 0 \\
		0 && 0 && 0 && -1
	\end{matrix}\right)
	\left(\begin{matrix}
		a && b_1 && b_2 && b_3 \\
		c_1 && D_{11} && D_{12} && D_{13} \\
		c_2 && D_{21} && D_{22} && D_{23}\\
		c_3 && D_{31} && D_{32} && D_{33}
	\end{matrix}\right) = H
\end{equation}

$$H = 	\left(\begin{matrix}
	a && b_1 && b_2 && b_3 \\
	-c_1 && -D_{11} && -D_{12} && -D_{13} \\
	-c_2 && -D_{21} && -D_{22} && -D_{23}\\
	-c_3 && -D_{31} && -D_{32} && -D_{33}
\end{matrix}\right)$$
\end{enumerate}
Por fim,

$$B = M^{T}H$$

calculando $ M^{T} $, temos

\begin{equation}\nonumber
	M^{T} = \left(\begin{matrix}
	a && c_1 && c_2 && c_3 \\
	b_1 && D_{11} && D_{21} && D_{31}\\
	b_2 && D_{12} && D_{22} && D_{32} \\
	b_3 && D_{13} && D_{23} && D_{33}
	\end{matrix}\right) 
\end{equation}

\begin{equation}\nonumber
	B = \left(\begin{matrix}
	a && c_1 && c_2 && c_3 \\
	b_1 && D_{11} && D_{21} && D_{31}\\
	b_2 && D_{12} && D_{22} && D_{32} \\
	b_3 && D_{13} && D_{23} && D_{33}
\end{matrix}\right) \left(\begin{matrix}
a && b_1 && b_2 && b_3 \\
-c_1 && -D_{11} && -D_{12} && -D_{13} \\
-c_2 && -D_{21} && -D_{22} && -D_{23}\\
-c_3 && -D_{31} && -D_{32} && -D_{33}
\end{matrix}\right)
\end{equation}

Calculando as componentes de $ B $, temos

\begin{eqnarray}
	B_{00} = a^2 -c_1^2 - c_2^2 - c_3^2 \nonumber\\
	B_{01} = ab_1 - D_{11}c_1 - D_{21}c_2 - D_{31}c_3 \nonumber\\
	B_{02} = ab_2 - D_{12}c_1 - D_{22}c_2 - D_{32}c_3 \nonumber\\
	B_{03} = ab_3 - D_{13}c_1 - D_{23}c_2 - D_{33}c_3 \nonumber\\
	B_{10} = ab_1 - D_{11}c_1 - D_{21}c_2 - D_{31}c_3 \nonumber\\
	B_{11} = b_1^2 - D_{11}^2 - D_{21}^2 - D_{31}^2 \nonumber\\
	B_{12} = b_1b_2 - D_{11}D_{12} - D_{21}D_{22} - D_{31}D_{32} \nonumber\\
	B_{13} = b_1b_3 - D_{11}D_{13} - D_{21}D_{23} - D_{31}D_{33} \nonumber\\
	B_{20} = ab_2 - D_{12}c_1 - D_{22}c_2 - D_{32}c_3 \nonumber\\
	B_{21} = b_1b_2 - D_{12}D_{11} - D_{22}D_{21} - D_{32}D_{31} \nonumber\\
	B_{22} = b_2^2 - D_{12}^2 - D_{22}^2 - D_{32}^2 \nonumber\\
	B_{23} = b_2b_3 - D_{12}D_{13} - D_{22}D_{23} - D_{32}D_{33} \nonumber\\
	B_{30} = ab_3 - D_{13}c_1 - D_{23}c_2 - D_{33}c_3 \nonumber\\
	B_{31} = b_1b_3 - D_{11}D_{13} - D_{21}D_{23} - D_{31}D_{33} \nonumber\\
	B_{32} = b_2b_3 - D_{12}D_{13} - D_{22}D_{23} - D_{32}D_{33} \nonumber\\
	B_{33} = b_3^2 - D_{13}^2 - D_{23}^2 - D_{33}^2 \nonumber
\end{eqnarray}

Se a matriz $ M $ respeita as condições do item anterior, então

\begin{eqnarray}
	B_{00} = a^2 -c_1^2 - c_2^2 - c_3^2 =1\nonumber\\
	B_{01} = ab_1 - D_{11}c_1 - D_{21}c_2 - D_{31}c_3=0 \nonumber\\
	B_{02} = ab_2 - D_{12}c_1 - D_{22}c_2 - D_{32}c_3=0 \nonumber\\
	B_{03} = ab_3 - D_{13}c_1 - D_{23}c_2 - D_{33}c_3=0 \nonumber\\
	B_{10} = ab_1 - D_{11}c_1 - D_{21}c_2 - D_{31}c_3=0 \nonumber\\
	B_{11} = b_1^2 - D_{11}^2 - D_{21}^2 - D_{31}^2=1 \nonumber\\
	B_{12} = b_1b_2 - D_{11}D_{12} - D_{21}D_{22} - D_{31}D_{32} =0\nonumber\\
	B_{13} = b_1b_3 - D_{11}D_{13} - D_{21}D_{23} - D_{31}D_{33}=0 \nonumber\\
	B_{20} = ab_2 - D_{12}c_1 - D_{22}c_2 - D_{32}c_3=0 \nonumber\\
	B_{21} = b_1b_2 - D_{12}D_{11} - D_{22}D_{21} - D_{32}D_{31}=0 \nonumber\\
	B_{22} = b_2^2 - D_{12}^2 - D_{22}^2 - D_{32}^2=1 \nonumber\\
	B_{23} = b_2b_3 - D_{12}D_{13} - D_{22}D_{23} - D_{32}D_{33} \nonumber\\
	B_{30} = ab_3 - D_{13}c_1 - D_{23}c_2 - D_{33}c_3 =0\nonumber\\
	B_{31} = b_1b_3 - D_{11}D_{13} - D_{21}D_{23} - D_{31}D_{33} =0\nonumber\\
	B_{32} = b_2b_3 - D_{12}D_{13} - D_{22}D_{23} - D_{32}D_{33}=0 \nonumber\\
	B_{33} = b_3^2 - D_{13}^2 - D_{23}^2 - D_{33}^2 =1 \nonumber
\end{eqnarray}
portanto, B é dado por

$$B = \left(\begin{matrix}
	1&&0&&0&&0\\
	0&&1&&0&&0\\
	0&&0&&1&&0\\
	0&&0&&0&&1
\end{matrix}\right)$$

e as condições do item (a) podem ser escritas por $ (M^T)_{\mu\alpha} (\eta)_{\alpha\beta} (M)_{\beta\nu }= \delta_{\mu\nu} $, pois obtemos as mesmas relações algébricas entre as componentes da matriz $ M $.

 %%%%%%%%%%%%%%%%%%%%%%%%%%%%%%%
 
 \section{Décimo Exercício}
Considere um fio onde corre uma corrente $I$. No referencial do laboratório, que vamos chamar de $S$, o fio está parado, é neutro (sem carga líquida) e se estende reto paralelo ao eixo $y$. Ainda para esse referencial, os elétrons responsáveis se deslocam pelo fio com velocidade constante $\mathbf{v}=v\mathbf{\hat{y}}$. Considere também uma carga $q$ a uma distância $R$ do fio. Considere ainda um referencial $\tilde{S}$ que se desloca, com relação à $S$, com velocidade $\mathbf{V}=v\mathbf{\hat{y}}$.

\begin{itemize}
\item[(a)] Determine os campos elétricos e magnéticos nos referenciais $S$ e $\tilde{S}$.

\paragraph{Resposta:} Primeiro vamos analisar o referencial $S$. Para ele, o fio está parado, porém a carga que passa no fio gera um campo magnético radial em torno do eixo $y$, dado por 

\begin{equation*}
		\mathbf{B} = \frac{\mu_0I}{2\pi s}\mathbf{\hat{s}} 
\end{equation*}

onde $s$ é a distância onde está sendo calculado o campo e $\mathbf{\hat{s}}$ um vetor unitário radial em torno de $y$. 

Como também esse referencial possui uma carga elétrica, ela gerará um campo elétrico radial dado por 

\begin{equation*}
\mathbf{E} = \frac{1}{4\pi \varepsilon_0}\frac{q}{r^2}\mathbf{\hat{r}}
\end{equation*}

onde $\mathbf{\hat{r}}$ é um vetor unitário na direção radial. 

Para o referencial $\tilde{S}$, haverá uma contração do fio na direção do movimento, ou seja, a densidade de cargas será maior, de forma que o fio \textbf{não estará mais neutro}, ou seja, sua carga líquida não será mais zero. Portanto, além do campo magnético gerado pela corrente elétrica e o campo elétrico da carga, o referencial $\tilde{S}$ vai verificar também um campo elétrico gerado pelo próprio fio, que agora não é mais neutro. Além disso, agora a carga elétrica também se move, de forma que ela produzirá um campo magnético. Portanto os campos são 

\begin{align*}
&\mathbf{\hat{B}}_{carga}=-\frac{\mu_0}{4\pi}\frac{qv(1-v^2/^2\sin\theta}{[1-(v^2/c^2)\sin^2\theta]^{3/2}}\frac{\mathbf{\tilde{\hat{s}}}}{s}\\
&\mathbf{\tilde{B}}_{fio} = -\frac{\mu_0I}{2\pi s}\mathbf{\tilde{\hat{s}}} \\
&\mathbf{\tilde{E}}_{carga} = \frac{1}{4\pi \varepsilon_0}\frac{q}{r^2}\mathbf{\tilde{\hat{r}}}\\
&\mathbf{\tilde{E}}_{fio} = \frac{1}{2\pi \varepsilon_0}\frac{\lambda}{r}\mathbf{\tilde{\hat{r}}}
\end{align*}

O sinal de menos para os campos magnéticos se da por conta da escolha do sentido vertical para cima como positivo. 
\item[(b)]Determine as forças que agem na carga q quando medidas por observadores nos referenciais $S$ e $\tilde{S}$. 

\paragraph{Resposta:} Novamente, vamos começar analisando para o referencial $S$. Nesse referencial, o campo magnético produzido pela corrente no fio realiza um força  sobre a carga. Onde, pela regra da mão esquerda, temos que essa força é dada por 

\begin{equation*}
	\mathbf{F} = q\mathbf{v}\times\mathbf{B} = q\mathbf{v}\times\left(\frac{\mu_0I}{2\pi R}\mathbf{\hat{s}}\right) = \left(\frac{\mu_0 q I}{2\pi R}\right)\mathbf{v}\times \mathbf{B} = \left(\frac{\mu_0 q I}{2\pi R}\right)\mathbf{\hat{x}}
\end{equation*}

Para o referencial $\tilde{S}$, como já analisamos antes, a corrente se da no sentido oposto, e surge uma densidade de cargas no fio. Portanto o fio exerce tanto uma força elétrica, quanto magnética sobre a carga. Dada por 

\begin{equation*}
\mathbf{\hat{F}} = q[\mathbf{\tilde{E}} + v\times \mathbf{\tilde{B}}] = q\left[\frac{1}{2\pi \varepsilon_0}\frac{\lambda}{r}\mathbf{\tilde{\hat{r}}} - \mathbf{\tilde{v}}\times\left(\frac{\mu_0I}{2\pi s}\mathbf{\tilde{\hat{s}}}\right)\right] = q\left[\frac{1}{2\pi \varepsilon_0}\frac{\lambda}{r}\mathbf{\tilde{\hat{r}}}-\left(\frac{\mu_0 I}{2\pi s}\right)\mathbf{\tilde{\hat{x}}}\right]
\end{equation*}

\end{itemize} 
 
 %%%%%%%%%%%%%%%%%%%%%%%%%%%%%%%
\section{Décimo Primeiro Exercício}
Mostre que duas transformações de Lorentz sucessivas na mesma direção, a primeira com velocidade $ v_1 $ e a segunda com velocidade $ v_2 $, equivalem
a uma única transformação de Lorentz, e calcule a velocidade $ v $ desta transformação. Discuta como esta velocidade resultante se relaciona com a fórmula de Einstein para a soma de velocidades. 

\paragraph{Resposta: }
A matriz de Tranformação de Lorentz (Boots na direção x) é dada por

$$ \Lambda = \left(\begin{matrix}
	\gamma&&-\gamma\beta&&0&&0\\
	-\gamma\beta&&\gamma&&0&&0\\
	0&&0&&1&&0\\
	0&&0&&0&&1
\end{matrix}\right)$$
 onde $\gamma = \left(1 -\frac{v^2}{c^2}\right)^{-1/2}$ e $\beta = \frac{v}{c}$.
 
 Podemos escrever a tranformação de Lorentz na forma matricial, como
 
 \begin{equation}\nonumber
 	\left(\begin{matrix}
 		x_0'\\
 		x_1'\\
 		x_2'\\
 		x_3'
 	\end{matrix}\right) = \left(\begin{matrix}
 	\gamma&&-\gamma\beta&&0&&0\\
 	-\gamma\beta&&\gamma&&0&&0\\
 	0&&0&&1&&0\\
 	0&&0&&0&&1
 \end{matrix}\right)\left(\begin{matrix}
 x_0\\
 x_1\\
 x_2\\
 x_3
\end{matrix}\right)
 \end{equation}
para a uma transformação (na direção x) com velocidade $ v_1 $ dado por $ \gamma(v_1) \equiv \gamma_1 $ e $ \beta(v_1) \equiv \beta_1 $, logo

 \begin{equation}\nonumber
	\left(\begin{matrix}
		x_0'\\
		x_1'\\
		x_2'\\
		x_3'
	\end{matrix}\right) = \left(\begin{matrix}
		\gamma_1&&-\gamma_1\beta_1&&0&&0\\
		-\gamma_1\beta_1&&\gamma_1&&0&&0\\
		0&&0&&1&&0\\
		0&&0&&0&&1
	\end{matrix}\right)\left(\begin{matrix}
		x_0\\
		x_1\\
		x_2\\
		x_3
	\end{matrix}\right)
\end{equation}
 em seguida se fizermos a segunda transformação com velocidade $ v_2 $ na mesma direção, isto é, $\gamma(v_2) \equiv \gamma_2$ e $\beta(v_2) = \beta_2$
 
  \begin{equation}\nonumber
 	\left(\begin{matrix}
 		x_0''\\
 		x_1''\\
 		x_2''\\
 		x_3''
 	\end{matrix}\right) = \left(\begin{matrix}
 		\gamma_2&&-\gamma_2\beta_2&&0&&0\\
 		-\gamma_2\beta_2&&\gamma_2&&0&&0\\
 		0&&0&&1&&0\\
 		0&&0&&0&&1
 	\end{matrix}\right)\left(\begin{matrix}
 	x_0'\\
 	x_1'\\
 	x_2'\\
 	x_3'
 \end{matrix}\right)
 \end{equation}
 
   \begin{equation}\nonumber
 	\left(\begin{matrix}
 		x_0''\\
 		x_1''\\
 		x_2''\\
 		x_3''
 	\end{matrix}\right) = \left(\begin{matrix}
 		\gamma_2&&-\gamma_2\beta_2&&0&&0\\
 		-\gamma_2\beta_2&&\gamma_2&&0&&0\\
 		0&&0&&1&&0\\
 		0&&0&&0&&1
 	\end{matrix}\right)\left(\begin{matrix}
 	\gamma_1&&-\gamma_1\beta_1&&0&&0\\
 	-\gamma_1\beta_1&&\gamma_1&&0&&0\\
 	0&&0&&1&&0\\
 	0&&0&&0&&1
 \end{matrix}\right)\left(\begin{matrix}
 x_0\\
 x_1\\
 x_2\\
 x_3
\end{matrix}\right)
 \end{equation}
 
 chamando essa dupla transformação de Lorentz de $\Lambda'$, temos
 
 \begin{equation}\nonumber
 	\Lambda' = \left(\begin{matrix}
 		\gamma_2&&-\gamma_2\beta_2&&0&&0\\
 		-\gamma_2\beta_2&&\gamma_2&&0&&0\\
 		0&&0&&1&&0\\
 		0&&0&&0&&1
 	\end{matrix}\right)\left(\begin{matrix}
 		\gamma_1&&-\gamma_1\beta_1&&0&&0\\
 		-\gamma_1\beta_1&&\gamma_1&&0&&0\\
 		0&&0&&1&&0\\
 		0&&0&&0&&1
 	\end{matrix}\right)
 \end{equation}
calculando a multiplicação de matrizes acima, obtemos


\begin{equation}\nonumber
	\Lambda' = \left(\begin{matrix}
		\gamma_2\gamma_1(1+\beta_1\beta_2)&&-\gamma_2\gamma_1(\beta_1+\beta_2)&&0&&0\\
		-\gamma_2\gamma_1(\beta_1+\beta_2)&&\gamma_2\gamma_1(1+\beta_1\beta_2)&&0&&0\\
		0&&0&&1&&0\\
		0&&0&&0&&1
	\end{matrix}\right)
\end{equation}

calculando explicitamente quem é $ \gamma_2\gamma_1(1+\beta_1\beta_2) $, temos

\begin{eqnarray}
	\gamma_2\gamma_1(1+\beta_1\beta_2) = \left(1 -\frac{v_1^2}{c^2}\right)^{-1/2}\left(1 -\frac{v_2^2}{c^2}\right)^{-1/2}(1+\beta_1\beta_2) = \left(1 -\beta_1^2\right)^{-1/2}\left(1 -\beta_2^2\right)^{-1/2}(1+\beta_1\beta_2) \nonumber 
\end{eqnarray}

$$\gamma_2\gamma_1(1+\beta_1\beta_2) =\left(1 -\beta_1^2\right)^{-1/2}\left(1 -\beta_2^2\right)^{-1/2}(1+\beta_1\beta_2)$$

$$\gamma_2\gamma_1(1+\beta_1\beta_2) =\left(\left(1 -\beta_1^2\right)\left(1 -\beta_2^2\right)\right)^{-1/2}(1+\beta_1\beta_2)$$

$$\gamma_2\gamma_1(1+\beta_1\beta_2) =\left(\left(1 -\beta_1\right)\left(1 +\beta_1\right)\left(1 -\beta_2\right)\left(1 -\beta_2\right)\right)^{-1/2}(1+\beta_1\beta_2)$$

$$\gamma_2\gamma_1(1+\beta_1\beta_2) =\left((1 -(\beta_2\beta_1)^2)^2\right)^{-1/2}(1+\beta_1\beta_2)$$

$$\gamma_2\gamma_1(1+\beta_1\beta_2) =(1 -(\beta_2\beta_1)^2)^{-1}(1+\beta_1\beta_2)= (1 -(\beta_2\beta_1))^{-1}(1 +(\beta_2\beta_1))^{-1}(1+\beta_1\beta_2)$$

$$\gamma_2\gamma_1(1+\beta_1\beta_2) = (1 -(\beta_2\beta_1))^{-1}$$

calculando $-\gamma_2\gamma_1(\beta_1+\beta_2)  $ explicitamente, temos

$$-\gamma_2\gamma_1(\beta_1+\beta_2)=- (1 -(\beta_2\beta_1)^2)^{-1} (\beta_1+\beta_2)$$

A partir daqui não tenho uma ideia clara de como resolver esse exercício.
Logo,

\begin{equation}\nonumber
	\Lambda' = \left(\begin{matrix}
		(1 -(\beta_2\beta_1))^{-1}&&-(1-(\beta_2\beta_1)^2)^{-1} (\beta_1+\beta_2)&&0&&0\\
		-(1-(\beta_2\beta_1)^2)^{-1} (\beta_1+\beta_2)&&(1 -(\beta_2\beta_1))^{-1}&&0&&0\\
		0&&0&&1&&0\\
		0&&0&&0&&1
	\end{matrix}\right)
\end{equation}
Segundo o enunciado, isso equivale a uma unica transformação de Lorentz dada por

$$\Lambda = \left(\begin{matrix}
	\gamma&&-\gamma\beta&&0&&0\\
	-\gamma\beta&&\gamma&&0&&0\\
	0&&0&&1&&0\\
	0&&0&&0&&1
\end{matrix}\right)$$
comparando os elementos da matriz, podemos encontrar a velocidade v, logo

$$\gamma = (1 -(\beta_2\beta_1))^{-1}$$

$$(1- \dfrac{v^2}{c^2})^{-1/2} = (1 -(\beta_2\beta_1))^{-1}$$

$$(1- \dfrac{v^2}{c^2})^{1/2} = 1 -\beta_2\beta_1$$

$$1- \dfrac{v^2}{c^2} = (1 -\beta_2\beta_1)^2$$

$$1- \dfrac{v^2}{c^2} = 1 -2\beta_2\beta_1+(\beta_2\beta_1)^2$$

$$- \dfrac{v^2}{c^2} =  -2\beta_2\beta_1+(\beta_2\beta_1)^2$$

$$\dfrac{v^2}{c^2} =  2\dfrac{v_2}{c}\dfrac{v_1}{c}+(\dfrac{v_2}{c}\dfrac{v_1}{c})^2$$

$$\dfrac{v^2}{c^2} =  2\dfrac{v_2}{c}\dfrac{v_1}{c}+(\dfrac{v_2}{c}\dfrac{v_1}{c})^2$$

$$v^2 =  2v_1v_2-\frac{1}{c^2}(v_1v_2)^2$$

$$v =  \sqrt{2v_1v_2-\frac{1}{c^2}(v_1v_2)^2}$$

\end{document}