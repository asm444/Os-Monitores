\documentclass[10pt,a4paper]{article}
\usepackage[utf8]{inputenc}
\usepackage[T1]{fontenc}
\usepackage{amsmath}
\usepackage{amssymb}
\usepackage{makeidx}
\usepackage{graphicx}
\usepackage[portuguese]{babel}
\author{Arthur de Souza Molina e Gabriel Capelini Magalhaes}
\title{Resolução da Lista e dos exercícios das notas.}
\begin{document}
	\maketitle
	\section{Primeiro Exercício}
	Qual a diferença entre os referenciais inerciais da Mecânica Clássica e os da Teoria da Relatividade Restrita?
	
	\paragraph{Resposta:}
	Na Mecânica Clássica, utilizamos as Transformações de Galileu como um dicionário para relacionar medidas entre referenciais inerciais. Na Teoria da Relatividade Restrita, utilizamos as Transformações de Lorentz para o mesmo propósito, uma vez que as TL satisfazem os postulados da TRR.
	
	%%%%%%%%%%%%%%%%%%%%%%%%%%%%
	
	\section{Terceiro Exercício}
	A distância até a estrela mais distante da nossa galáxia é da ordem de $10^{5}$ anos-luz. Explique por que é possível, em princípio, para um ser humano viajar para tal estrela durante seu tempo de vida (digamos 80 anos) e faça uma estimativa da velocidade necessária para isso.
	\paragraph{Resposta:}
	Antes de começar qualquer cálculo, perceba que as medidas de intervalo de tempo e de distância são de um observador em repouso no referencial da Terra. Este problema é análogo ao paradoxo dos gêmeos.
	
	Podemos definir o observador na Terra como o referencial S ( o observador na estrela está em repouso em relação ao referencial S) e o observador (o viajante) da nave em repouso em seu respectivo referencial S' que se distância da Terra com a velocidade iremos estimar, isto é, a velocidade relativa entre os referenciais S e S', lembrando que o sentido da velocidade do foguete se altera na mudança de referencial, logo,
	
	\begin{equation}\label{v_foguete}
		  v' = \dfrac{\Delta x'}{\Delta t'}
	\end{equation}
	
	Sabendo que o observador na Terra realiza medidas de comprimento contraídas e intervalos de tempo dilatadas
	\begin{equation}
		\Delta x \approx 10^5 \text{  anos-luz    e  } \Delta t \approx \text{80 anos},
	\end{equation}
	podemos usar as TL para traduzir as medidas para o sistema de coordenadas para S' com
	\begin{equation}\label{TL_repouso}
		\Delta x' = \dfrac{\Delta x}{\gamma}\text{  e  } \Delta t' = \gamma\Delta t.
	\end{equation}

	Ao substituirmos temos
	\begin{eqnarray}
		&& v' = \dfrac{\dfrac{\Delta x}{\gamma}}{\gamma\Delta t} \nonumber \\
		&& v' = \dfrac{\Delta x}{\gamma^2\Delta t}\nonumber
	\end{eqnarray}
	para facilitar as contas, podemos realizar a seguinte operação
	
	\begin{equation}\nonumber
		\dfrac{\Delta x}{\Delta t } = \dfrac{10^5 \text{  anos-luz}}{80\text{ anos}} = \dfrac{10^5c}{80} = 1,25\cdot10^3c
	\end{equation}
	irei definir $ \alpha\equiv 1,25\cdot10^3 $, apenas para não carregar um termo numérico imenso e dar o trabalho de manipula-lo. Retornando as contas temos
	
	\begin{eqnarray}
	    && v' = \dfrac{\alpha c}{\gamma^2}\nonumber \\
		&& v' = \alpha c \left( 1 - \dfrac{v'^2}{c^2}\right) \nonumber \\
		&& v'  =  \alpha c - \dfrac{ \alpha v'^2}{c} \nonumber \\		
		&& \dfrac{ c v'}{\alpha} = c^2 - v'^2 \nonumber \\
		&& v'^2 - \dfrac{ c v'}{\alpha} - c^2 = 0, \nonumber
	\end{eqnarray}
	usando Baskara, temos
	\begin{eqnarray}
		&& v' = -\dfrac{\frac{c}{\alpha} \pm \sqrt{\left(\frac{c}{\alpha}\right)^2 +4c^2}}{2} \nonumber \\
		&& v' = \dfrac{-\frac{c}{\alpha} \pm c\sqrt{\frac{1}{\alpha^2} +4}}{2} \nonumber 
	\end{eqnarray}
	como isso é uma estimativa, podemos considerar que $ \alpha^{-2} \approx 0 $, logo
	\begin{eqnarray}
		&& v' = \dfrac{-\frac{c}{\alpha} \pm c\sqrt{4}}{2} \nonumber \\
		&& v' = c\dfrac{-\frac{1}{\alpha} \pm 2}{2} \nonumber\\
		&& v' = c\left(-\frac{1}{2\alpha} \pm 1\right) \nonumber 
	\end{eqnarray}
	O primeiro resultado, o negativo, temos
	$ v' = c\left(-\frac{1}{2\alpha} - 1\right) \Longrightarrow |v'| > |c|$, ultrapassando a velocidade da luz, na direção contrária da viagem e isso é incorreto.
	
	Já o segundo resultado, o positivo, temos
	$ v' = c\left(-\frac{1}{2\alpha} + 1\right) \approx 0,9996 c$ que é o resultado correto.

	%%%%%%%%%%%%%%%%%%%%%%

	\section{Quinto Exercício}
	Uma barra de comprimento próprio $ L_0 $ (i.e., o comprimento medido no referencial $ S_0 $ em que ela está em repouso) faz um ângulo $ \theta_0 $ com o eixo
	horizontal de seu referencial. Para um observador em um referencial S que se desloca com relação à $ S_0 $ na direção horizontal com velocidade v:
	\begin{enumerate}
		\item Qual o valor do ângulo $ \theta $ que a barra faz com o eixo horizontal?
		
		\paragraph{Resposta:}
		Como $ S_0 $ se desloca com velocidade v na horizontal, isto é, no eixo x, portanto a componente vertical da barra ficará ilesa, pois a contração espacial ocorre apenas da direção do movimento.
		
		\begin{eqnarray}
			&& \theta_0 = \arctan\left(\frac{\Delta y_0}{\Delta x_0}\right) = \arctan\left(\frac{L_y}{L_x}\right)\nonumber \\
			&& L_0 = \sqrt{L_y^2+L_x^2} \nonumber\\
			&& \theta = \arctan\left(\frac{\Delta y}{\Delta x}\right) \nonumber \\
		\end{eqnarray}
		 Reescrevendo $ \frac{L_y}{L_x} \equiv \alpha$ em termos de $ \theta_0 $, temos
		 $$\theta_0 = \arctan\left(\frac{\Delta y_0}{\Delta x_0}\right) = \arctan\left(\frac{L_y}{L_x}\right) =  \arctan (\alpha)$$
		 
		 $$\therefore \alpha = \tan (\theta_0)$$
		 
		 Sabendo que $ \Delta y = \Delta y_0 = L_y $ e $\Delta x = \dfrac{\Delta x_0}{\gamma} =\dfrac{L_x}{\gamma} $, temos
		 
		 $$ \theta = \arctan\left(\frac{\Delta y}{\Delta x}\right) = \arctan\left(\frac{\Delta y_0}{ \dfrac{\Delta x_0}{\gamma}}\right) =\arctan\left(\frac{\Delta y_0}{ \dfrac{\Delta x_0}{\gamma}}\right) = \arctan(\gamma\alpha)$$
		 
		 $$\theta = \arctan(\gamma\alpha) = \arctan(\gamma\tan (\theta_0))$$
		 
		$$ \therefore \theta  = \arctan(\gamma\tan (\theta_0))$$
		
		\item Qual o comprimento L da barra em S?
		 \paragraph{Resposta:}
		 O comprimento da barra em $ S_0 $ é dado por
		 $$ L_0 = \sqrt{L_y^2+L_x^2}$$
		 e para o comprimento da barra em $ S $ é dado por
		 $$ L= \sqrt{L_y'^2+L_x'^2}$$
		 
		 $$ L= \sqrt{L_y^2+\left(\dfrac{L_x}{\gamma}\right)^2}$$
		 
		 $$ L= \sqrt{(L_x)^2\left(\left(\dfrac{L_y}{L_x}\right)^2+\left(\dfrac{1}{\gamma}\right)^2\right)} = L_x \sqrt{(\alpha)^2+\left(\dfrac{1}{\gamma}\right)^2} = L_x \sqrt{\tan^2 (\theta_0)+ 1 - 1 + \left(\dfrac{1}{\gamma}\right)^2} $$
		 
		 $$ L= L_x \sqrt{\sec^2 \theta_0 - 1 + \left(1 - \frac{v^2}{c^2}\right)} = L_x \sqrt{\sec^2 \theta_0 - \frac{v^2}{c^2}}$$
		 
		 podemos escrever $ L_x = L_0 \cos\theta_0$, ao substituir
		 
		 $$L=  L_x  \sqrt{\sec^2 \theta_0 - \frac{v^2}{c^2}} = L_0 \cos\theta_0\sqrt{\sec^2 \theta_0 - \frac{v^2}{c^2}} = L_0 \sqrt{\cos^2\theta_0\sec^2 \theta_0 - \frac{v^2\cos^2\theta_0}{c^2}}  $$
		 
		 $$\therefore L= L_0 \sqrt{1 - \frac{v^2\cos^2\theta_0}{c^2}}$$
	\end{enumerate}
	
	%%%%%%%%%%%%%%
	
	\section{Sétimo Exercício}
	Num certo referencial S um observador registra a seguinte sequência de eventos: num certo instante uma bomba explodiu numa certa posição e, 3 segundo depois, uma segunda bomba explodiu a uma distância de 1 metro
	desta primeira bomba. É possível encontrar um referencial inercial $ S_0 $,que respeite todos os postulados da relatividade, onde estas duas bombas explodiram no mesmo instante? Se sim, qual a velocidade deste referencial
	em relação a S. Se não, justifique sua resposta.
	
	\paragraph{Resposta:}
	Temos as medidas realizadas no referencial S
	
	 $$\Delta x = 1 m \,\,\,\, \Delta t = 3s $$
	 
	 precisamos encontrar um referencial S' em que um observador registra as duas explosões ocorrendo simultaneamente ($ \Delta t' = 0 $), ao usar uma das transformações de Lorentz podemos encontrar a velocidade relativa entre os referenciais S e S'.
	 
	 $$ \Delta t' = \gamma \left( \Delta t - v \dfrac{\Delta x}{c^2}\right)$$

	 $$ 0 = \gamma \left( 3 - v \dfrac{1}{c^2}\right) $$
	 
	 $$ 0 =  3 - v \dfrac{1}{c^2} $$
	 
	 $$ v \dfrac{1}{c^2} = 3$$
	 
	 $$ v = 3c^2$$
	O referencial S' precisaria estar a uma velocidade relativa ao referencial S de $ 3c^2 $, portanto não é encontrar um referencial que consiga registrar os dois eventos ocorrendo simultaneamente.
	
	Outra maneira de resolver esse problema é verificar qual tipo de distância que temos entre os eventos
	
	$$( (\Delta s)^2 = \Delta x)^2 - c^2 (\Delta t)^2 = 1^2 -9c^2 < 0 $$
	portanto, a distância entre os evento é do tipo-tempo, ou seja, há uma relação de causalidade entre os eventos, logo é impossível encontrar um referencial que consiga ver os dois eventos ocorrerem simultaneamente.
	
	%%%%%%%%%%%%%%%%%%%%%%
	
	\section{Nono Exercício}
	Num sistema onde nomeamos as coordenadas como
	$$x_0 = ct,\,\,x_1 =x,\,\,x_2=y,\,\,x_3=z$$
	considere a seguinte transformação linear
	$$x_\mu = \sum_{\nu=0}^{3} M_{\mu\nu} x_\nu,\,\,\,\mu =0,1,2,3.$$
	\begin{enumerate}
		\item Determine as condições que $ M_{\mu\nu} $ deve respeitar para que
		
		$$ \text{\~{x}}^2_0 - \sum_{i=1}^{3} \text{\~{x}}^2_i = \text{x}^2_0 - \sum_{i=1}^{3} \text{x}^2_i $$
	
	\paragraph{Resposta:}
	Abrindo as componentes da transformação linear temos
	
	$$ x_\mu = M_{\mu0} x_0 + M_{\mu1} x_1+M_{\mu2} x_2+M_{\mu3} x_3 $$
	para $ \mu =0 $
	$$ \text{\~{x}}_0 = M_{00} x_0 + M_{01} x_1+M_{02} x_2+M_{03} x_3 $$
	para $ \mu =1 $
	$$ \text{\~{x}}_1 = M_{10} x_0 + M_{11} x_1+M_{12} x_2+M_{13} x_3 $$
	para $ \mu =2 $
	$$\text{\~{x}}_2 = M_{20} x_0 + M_{21} x_1+M_{22} x_2+M_{23} x_3 $$
	para $ \mu =3 $
	$$ \text{\~{x}}_3 = M_{30} x_0 + M_{31} x_1+M_{32} x_2+M_{33} x_3 $$
	 
	 calculando o $( \text{\~{x}}_\mu)^2 $ para substituir na equação acima.
	 
	 Para cada $ \mu=0 $
	 $$ (\text{\~{x}}_0)^2 =  (M_{00} x_0 + M_{01} x_1+M_{02} x_2+M_{03} x_3)^2 $$
	 \begin{equation}\nonumber
	 	\begin{split}
	 		(\text{\~{x}}_0)^2 =  M_{00}^2 x_0^2 + M_{00}M_{01} x_0x_1+M_{02}M_{00} x_0 x_2+M_{03}M_{00} x_0 x_3\\ 
	 		M_{00}M_{01} x_1 x_0 + M_{01}^2 x_1^2+M_{02}M_{01} x_1 x_2+M_{03}M_{01} x_1 x_3\\
	 		M_{00}M_{02} x_2 x_0 + M_{01}M_{02} x_2 x_1+M_{02}^2x_2^2+M_{03} M_{02} x_2x_3\\
	 		M_{00}M_{03} x_3 x_0 + M_{01}M_{03} x_3 x_1+M_{02}M_{03} x_3 x_2+M_{03}^2 x_3^2
	 	\end{split}
	 \end{equation}
 
	 \begin{equation}\nonumber
	 	\begin{split}
	 		(\text{\~{x}}_0)^2 =  M_{00}^2 x_0^2 +M_{01}^2 x_1^2+M_{02}^2x_2^2+M_{03}^2 x_3^2+ 2M_{00}M_{01} x_0x_1+2M_{02}M_{00} x_0 x_2\\+2M_{01}M_{03} x_3 x_1+2M_{03}M_{00} x_0 x_3+2M_{02}M_{01} x_1 x_2+2M_{02}M_{03} x_3 x_2
	 	\end{split}
	 \end{equation}
 
 	para $ \mu =1 $
 	$$ (\text{\~{x}}_1)^2 = (M_{10} x_0 + M_{11} x_1+M_{12} x_2+M_{13} x_3)^2 $$
 	\begin{equation}\nonumber
 		\begin{split}
 			(\text{\~{x}}_1)^2 = M_{10}^2 x_0^2 + M_{11}M_{10} x_0 x_1+M_{12} M_{10} x_0x_2+M_{13}M_{10} x_0 x_3\\
 			M_{10}M_{11} x_1 x_0 + M_{11}^2 x_1^2+M_{12}M_{11} x_1 x_2+M_{13}M_{11} x_1 x_3\\
 			M_{10}M_{12} x_2 x_0 + M_{11}M_{12} x_2 x_1+M_{12}^2 x_2^2+M_{13}M_{12} x_2 x_3\\
 			M_{10}M_{13} x_3x_0 + M_{11}M_{13} x_3 x_1+M_{12} M_{13} x_3x_2+M_{13}^2 x_3^2
 		\end{split}
 	\end{equation}
 
 	\begin{equation}\nonumber
 		\begin{split}
 			(\text{\~{x}}_1)^2 = M_{10}^2 x_0^2 + M_{11}^2+M_{12}^2 x_2^2+M_{13}^2 x_3^2+2M_{11}M_{10} x_0 x_1+2M_{12} M_{10} x_0x_2\\+2M_{13}M_{10} x_0 x_3+2M_{11}M_{13} x_3 x_1+2M_{12} M_{13} x_3x_2+2M_{12}M_{11} x_1 x_2
 		\end{split}
 	\end{equation}
	
	para $ \mu =2 $
	$$( \text{\~{x}}_2)^2 = (M_{20} x_0 + M_{21} x_1+M_{22} x_2+M_{23} x_3)^2 $$
	
	\begin{equation}\nonumber
		\begin{split}
			( \text{\~{x}}_2)^2 = M_{20}^2 x_0^2 + M_{21}M_{20} x_0 x_1+M_{22} M_{20} x_0x_2+M_{23} M_{20} x_0x_3\\
			M_{20}M_{21} x_1 x_0 + M_{21}^2 x_1^2+M_{22} M_{21} x_1x_2+M_{23}M_{21} x_1 x_3\\
			M_{20} x_0 M_{22} x_2+ M_{21} M_{22} x_2x_1+M_{22}^2 x_2^2+M_{23} M_{22} x_2x_3\\
			M_{20}M_{23} x_3 x_0 + M_{21} M_{23} x_3x_1+M_{22}M_{23} x_3 x_2+M_{23}^2 x_3^2	
		\end{split}
	\end{equation}

\begin{equation}\nonumber
	\begin{split}
		( \text{\~{x}}_2)^2 = M_{20}^2 x_0^2 + M_{21}^2 x_1^2+M_{22}^2 x_2^2+M_{23}^2 x_3^2	+2 M_{21}M_{20} x_0 x_1+2M_{22} M_{20}
		 x_0x_2\\+2M_{23} M_{20} x_0x_3+2M_{21} M_{23} x_3x_1+2M_{22}M_{23} x_3 x_2+2 M_{21} M_{22} x_2x_1
	\end{split}
\end{equation}

	para $ \mu =3 $
	$$ (\text{\~{x}}_3)^2 =( M_{30} x_0 + M_{31} x_1+M_{32} x_2+M_{33} x_3)^2 $$
	
	\begin{equation}\nonumber
		\begin{split}
			(\text{\~{x}}_3)^2 =M_{30}^2 x_0^2 + M_{31}M_{30} x_0 x_1+M_{32} M_{30} x_0x_2+M_{33}M_{30} x_0 x_3\\
			M_{30}M_{31} x_1 x_0 + M_{31}^2 x_1^2+M_{32}M_{31} x_1 x_2+M_{33}M_{31} x_1 x_3\\
			M_{30}M_{32} x_2 x_0 + M_{31} M_{32} x_2x_1+M_{32}^2 x_2^2+M_{33}M_{32} x_2 x_3\\
			 M_{30}M_{33} x_3 x_0 + M_{31} M_{33} x_3x_1+M_{32}M_{33} x_3 x_2+M_{33}^2 x_3^2
		\end{split}
	\end{equation}

		\begin{equation}\nonumber
		\begin{split}
			(\text{\~{x}}_3)^2 =M_{30}^2 x_0^2 + M_{31}^2 x_1^2+M_{32}^2 x_2^2+M_{33}^2 x_3^2+2M_{31}M_{30} x_0 x_1+2M_{32} M_{30} x_0x_2\\+2M_{33}M_{30} x_0 x_3+2 M_{31} M_{33} x_3x_1+2M_{32}M_{33} x_3 x_2+2M_{33}M_{32} x_2 x_3
		\end{split}
	\end{equation}
	
	subsituindo, temos
	\begin{equation}\nonumber
		\begin{split}
			M_{00}^2 x_0^2 +M_{01}^2 x_1^2+M_{02}^2x_2^2+M_{03}^2 x_3^2+ 2M_{00}M_{01} x_0x_1+2M_{02}M_{00} x_0 x_2\\+2M_{01}M_{03} x_3 x_1+2M_{03}M_{00} x_0 x_3+2M_{02}M_{01} x_1 x_2+2M_{02}M_{03} x_3 x_2 \\
			-(M_{10}^2 x_0^2 + M_{11}^2+M_{12}^2 x_2^2+M_{13}^2 x_3^2+2M_{11}M_{10} x_0 x_1+2M_{12} M_{10} x_0x_2\\+2M_{13}M_{10} x_0 x_3+2M_{11}M_{13} x_3 x_1+2M_{12} M_{13} x_3x_2+2M_{12}M_{11} x_1 x_2\\
			M_{20}^2 x_0^2 + M_{21}^2 x_1^2+M_{22}^2 x_2^2+M_{23}^2 x_3^2	+2 M_{21}M_{20} x_0 x_1+2M_{22} M_{20}
			x_0x_2\\+2M_{23} M_{20} x_0x_3+2M_{21} M_{23} x_3x_1+2M_{22}M_{23} x_3 x_2+2 M_{21} M_{22} x_2x_1\\
			M_{30}^2 x_0^2 + M_{31}^2 x_1^2+M_{32}^2 x_2^2+M_{33}^2 x_3^2+2M_{31}M_{30} x_0 x_1+2M_{32} M_{30} x_0x_2\\+2M_{33}M_{30} x_0 x_3+2 M_{31} M_{33} x_3x_1+2M_{32}M_{33} x_3 x_2+2M_{33}M_{32} x_2 x_3) \\
			= \text{x}^2_0 -  \text{x}^2_1 -  \text{x}^2_2 -  \text{x}^2_3
		\end{split}
	\end{equation}
	Agrupando os termos e relacionando com os termos do lado direito da equação, obtemos
	
	\begin{eqnarray}
		&& (M_{00}^2-M_{10}^2-M_{20}^2-M_{30}^2)x^2_0 = x^2_0\nonumber \\
		&& (M_{01}^2-M_{11}^2-M_{21}^2-M_{31}^2)x^2_1 = x^2_1\nonumber \\
		&& (M_{02}^2-M_{12}^2-M_{22}^2-M_{32}^2)x^2_2 = x^2_2\nonumber \\
		&& (M_{03}^2-M_{13}^2-M_{23}^2-M_{33}^2)x^2_3 = x^2_3\nonumber \\
		&& 2(M_{00}M_{01}-M_{11}M_{10}-M_{21}M_{20}-M_{31}M_{30})x_0x_1 = 0 x_0x_1\nonumber \\
		&&2(M_{00}M_{02}- M_{12}M_{10}-M_{22}M_{20}-M_{32}M_{30})x_0x_2 = 0 x_0x_2\nonumber \\
		&&2(M_{00}M_{03}- M_{13}M_{10}-M_{23}M_{20}-M_{33}M_{30})x_0x_3 = 0 x_0x_3\nonumber \\
		&&2(M_{01}M_{02}- M_{11}M_{12}-M_{21}M_{22}- M_{31}M_{32})x_1x_2 = 0x_1x_2\nonumber \\
		&& 2(M_{01}M_{03}-M_{11}M_{13}-M_{21}M_{23}-M_{31}M_{33})x_1x_3 = 0 x_1x_3\nonumber \\
		&& 2(M_{02}M_{03}-M_{12}M_{13}-M_{22}M_{23}-M_{32}M_{33})x_2x_3 = 0 x_2x_3 \nonumber
	\end{eqnarray}

	As condições que $ M_{\mu\nu} $ deve respeitar são
	\begin{eqnarray}
		&& M_{00}^2-M_{10}^2-M_{20}^2-M_{30}^2 = 1\nonumber \\
		&&M_{01}^2-M_{11}^2-M_{21}^2-M_{31}^2 = 1\nonumber \\
		&&M_{02}^2-M_{12}^2-M_{22}^2-M_{32}^2 = 1 \nonumber \\
		&& M_{03}^2-M_{13}^2-M_{23}^2-M_{33}^2 = 1\nonumber \\
		&& M_{00}M_{01}-M_{11}M_{10}-M_{21}M_{20}-M_{31}M_{30} = 0 \nonumber \\
		&&M_{00}M_{02}- M_{12}M_{10}-M_{22}M_{20}-M_{32}M_{30}= 0 \nonumber \\
		&&M_{00}M_{03}- M_{13}M_{10}-M_{23}M_{20}-M_{33}M_{30} = 0 \nonumber \\
		&&M_{01}M_{02}- M_{11}M_{12}-M_{21}M_{22}- M_{31}M_{32} = 0\nonumber \\
		&& M_{01}M_{03}-M_{11}M_{13}-M_{21}M_{23}-M_{31}M_{33}= 0 \nonumber \\
		&& M_{02}M_{03}-M_{12}M_{13}-M_{22}M_{23}-M_{32}M_{33} = 0  \nonumber
	\end{eqnarray}
	
	\item Agora escreva
	\begin{equation}\nonumber
		M =\left(\begin{matrix}
			a && \mathbf{b}^{T} \\
			\mathbf{c} && D
		\end{matrix}\right)
	\end{equation}
com $ a $ um número $ D $ uma matriz $ 3 \times 3 $ e
\begin{equation}\nonumber
	b = \left(\begin{matrix}
		b1 \\
		b2 \\
		b3
	\end{matrix}\right) \,\,\, ,	b = \left(\begin{matrix}
	c1 \\
	c2 \\
	c3
\end{matrix}\right).
\end{equation}
Mostre que as condições obtidas no item anterior podem ser escritas como
\begin{equation}\nonumber
	M^T\eta M = B \text{   onde } \eta = \left( \begin{matrix}
		1 && 0 \\
		0 && -I
	\end{matrix}\right).
\end{equation}
\paragraph{Resposta:}
Calculando o produto matricial $ \eta M $, obtemos

\begin{equation}\nonumber
	M = \left(\begin{matrix}
		a && b_1 && b_2 && b_3 \\
		c_1 && D_{11} && D_{12} && D_{13} \\
		c_2 && D_{21} && D_{22} && D_{23}\\
		c_3 && D_{31} && D_{32} && D_{33}
	\end{matrix}\right) 
\end{equation}

\begin{equation}\nonumber
	\left(\begin{matrix}
		1 && 0 && 0 && 0 \\
		0 && -1 && 0 && 0 \\
		0 && 0 && -1 && 0 \\
		0 && 0 && 0 && -1
	\end{matrix}\right)
	\left(\begin{matrix}
		a && b_1 && b_2 && b_3 \\
		c_1 && D_{11} && D_{12} && D_{13} \\
		c_2 && D_{21} && D_{22} && D_{23}\\
		c_3 && D_{31} && D_{32} && D_{33}
	\end{matrix}\right) = H
\end{equation}

$$H = 	\left(\begin{matrix}
	a && b_1 && b_2 && b_3 \\
	-c_1 && -D_{11} && -D_{12} && -D_{13} \\
	-c_2 && -D_{21} && -D_{22} && -D_{23}\\
	-c_3 && -D_{31} && -D_{32} && -D_{33}
\end{matrix}\right)$$
\end{enumerate}
Por fim,

$$B = M^{T}H$$

calculando $ M^{T} $, temos

\begin{equation}\nonumber
	M^{T} = \left(\begin{matrix}
	a && c_1 && c_2 && c_3 \\
	b_1 && D_{11} && D_{21} && D_{31}\\
	b_2 && D_{12} && D_{22} && D_{32} \\
	b_3 && D_{13} && D_{23} && D_{33}
	\end{matrix}\right) 
\end{equation}

\begin{equation}\nonumber
	B = \left(\begin{matrix}
	a && c_1 && c_2 && c_3 \\
	b_1 && D_{11} && D_{21} && D_{31}\\
	b_2 && D_{12} && D_{22} && D_{32} \\
	b_3 && D_{13} && D_{23} && D_{33}
\end{matrix}\right) \left(\begin{matrix}
a && b_1 && b_2 && b_3 \\
-c_1 && -D_{11} && -D_{12} && -D_{13} \\
-c_2 && -D_{21} && -D_{22} && -D_{23}\\
-c_3 && -D_{31} && -D_{32} && -D_{33}
\end{matrix}\right)
\end{equation}

Calculando as componentes de $ B $, temos

\begin{eqnarray}
	B_{00} = a^2 -c_1^2 - c_2^2 - c_3^2 \nonumber\\
	B_{01} = ab_1 - D_{11}c_1 - D_{21}c_2 - D_{31}c_3 \nonumber\\
	B_{02} = ab_2 - D_{12}c_1 - D_{22}c_2 - D_{32}c_3 \nonumber\\
	B_{03} = ab_3 - D_{13}c_1 - D_{23}c_2 - D_{33}c_3 \nonumber\\
	B_{10} = ab_1 - D_{11}c_1 - D_{21}c_2 - D_{31}c_3 \nonumber\\
	B_{11} = b_1^2 - D_{11}^2 - D_{21}^2 - D_{31}^2 \nonumber\\
	B_{12} = b_1b_2 - D_{11}D_{12} - D_{21}D_{22} - D_{31}D_{32} \nonumber\\
	B_{13} = b_1b_3 - D_{11}D_{13} - D_{21}D_{23} - D_{31}D_{33} \nonumber\\
	B_{20} = ab_2 - D_{12}c_1 - D_{22}c_2 - D_{32}c_3 \nonumber\\
	B_{21} = b_1b_2 - D_{12}D_{11} - D_{22}D_{21} - D_{32}D_{31} \nonumber\\
	B_{22} = b_2^2 - D_{12}^2 - D_{22}^2 - D_{32}^2 \nonumber\\
	B_{23} = b_2b_3 - D_{12}D_{13} - D_{22}D_{23} - D_{32}D_{33} \nonumber\\
	B_{30} = ab_3 - D_{13}c_1 - D_{23}c_2 - D_{33}c_3 \nonumber\\
	B_{31} = b_1b_3 - D_{11}D_{13} - D_{21}D_{23} - D_{31}D_{33} \nonumber\\
	B_{32} = b_2b_3 - D_{12}D_{13} - D_{22}D_{23} - D_{32}D_{33} \nonumber\\
	B_{33} = b_3^2 - D_{13}^2 - D_{23}^2 - D_{33}^2 \nonumber
\end{eqnarray}

Se a matriz $ M $ respeita as condições do item anterior, então

\begin{eqnarray}
	B_{00} = a^2 -c_1^2 - c_2^2 - c_3^2 =1\nonumber\\
	B_{01} = ab_1 - D_{11}c_1 - D_{21}c_2 - D_{31}c_3=0 \nonumber\\
	B_{02} = ab_2 - D_{12}c_1 - D_{22}c_2 - D_{32}c_3=0 \nonumber\\
	B_{03} = ab_3 - D_{13}c_1 - D_{23}c_2 - D_{33}c_3=0 \nonumber\\
	B_{10} = ab_1 - D_{11}c_1 - D_{21}c_2 - D_{31}c_3=0 \nonumber\\
	B_{11} = b_1^2 - D_{11}^2 - D_{21}^2 - D_{31}^2=1 \nonumber\\
	B_{12} = b_1b_2 - D_{11}D_{12} - D_{21}D_{22} - D_{31}D_{32} =0\nonumber\\
	B_{13} = b_1b_3 - D_{11}D_{13} - D_{21}D_{23} - D_{31}D_{33}=0 \nonumber\\
	B_{20} = ab_2 - D_{12}c_1 - D_{22}c_2 - D_{32}c_3=0 \nonumber\\
	B_{21} = b_1b_2 - D_{12}D_{11} - D_{22}D_{21} - D_{32}D_{31}=0 \nonumber\\
	B_{22} = b_2^2 - D_{12}^2 - D_{22}^2 - D_{32}^2=1 \nonumber\\
	B_{23} = b_2b_3 - D_{12}D_{13} - D_{22}D_{23} - D_{32}D_{33} \nonumber\\
	B_{30} = ab_3 - D_{13}c_1 - D_{23}c_2 - D_{33}c_3 =0\nonumber\\
	B_{31} = b_1b_3 - D_{11}D_{13} - D_{21}D_{23} - D_{31}D_{33} =0\nonumber\\
	B_{32} = b_2b_3 - D_{12}D_{13} - D_{22}D_{23} - D_{32}D_{33}=0 \nonumber\\
	B_{33} = b_3^2 - D_{13}^2 - D_{23}^2 - D_{33}^2 =1 \nonumber
\end{eqnarray}
portanto, B é dado por

$$B = \left(\begin{matrix}
	1&&0&&0&&0\\
	0&&1&&0&&0\\
	0&&0&&1&&0\\
	0&&0&&0&&1
\end{matrix}\right)$$

e as condições do item (a) podem ser escritas por $ (M^T)_{\mu\alpha} (\eta)_{\alpha\beta} (M)_{\beta\nu }= \delta_{\mu\nu} $, pois obtemos as mesmas relações algébricas entre as componentes da matriz $ M $.

\section{Décimo Primeiro Exercício}
Mostre que duas transformações de Lorentz sucessivas na mesma direção, a primeira com velocidade $ v_1 $ e a segunda com velocidade $ v_2 $, equivalem
a uma única transformação de Lorentz, e calcule a velocidade $ v $ desta transformação. Discuta como esta velocidade resultante se relaciona com a fórmula de Einstein para a soma de velocidades. 

\paragraph{Resposta: }
A matriz de Tranformação de Lorentz (Boots na direção x) é dada por

$$ \Lambda = \left(\begin{matrix}
	\gamma&&-\gamma\beta&&0&&0\\
	-\gamma\beta&&\gamma&&0&&0\\
	0&&0&&1&&0\\
	0&&0&&0&&1
\end{matrix}\right)$$
 onde $\gamma = \left(1 -\frac{v^2}{c^2}\right)^{-1/2}$ e $\beta = \frac{v}{c}$.
 
 Podemos escrever a tranformação de Lorentz na forma matricial, como
 
 \begin{equation}\nonumber
 	\left(\begin{matrix}
 		x_0'\\
 		x_1'\\
 		x_2'\\
 		x_3'
 	\end{matrix}\right) = \left(\begin{matrix}
 	\gamma&&-\gamma\beta&&0&&0\\
 	-\gamma\beta&&\gamma&&0&&0\\
 	0&&0&&1&&0\\
 	0&&0&&0&&1
 \end{matrix}\right)\left(\begin{matrix}
 x_0\\
 x_1\\
 x_2\\
 x_3
\end{matrix}\right)
 \end{equation}
para a uma transformação (na direção x) com velocidade $ v_1 $ dado por $ \gamma(v_1) \equiv \gamma_1 $ e $ \beta(v_1) \equiv \beta_1 $, logo

 \begin{equation}\nonumber
	\left(\begin{matrix}
		x_0'\\
		x_1'\\
		x_2'\\
		x_3'
	\end{matrix}\right) = \left(\begin{matrix}
		\gamma_1&&-\gamma_1\beta_1&&0&&0\\
		-\gamma_1\beta_1&&\gamma_1&&0&&0\\
		0&&0&&1&&0\\
		0&&0&&0&&1
	\end{matrix}\right)\left(\begin{matrix}
		x_0\\
		x_1\\
		x_2\\
		x_3
	\end{matrix}\right)
\end{equation}
 em seguida se fizermos a segunda transformação com velocidade $ v_2 $ na mesma direção, isto é, $\gamma(v_2) \equiv \gamma_2$ e $\beta(v_2) = \beta_2$
 
  \begin{equation}\nonumber
 	\left(\begin{matrix}
 		x_0''\\
 		x_1''\\
 		x_2''\\
 		x_3''
 	\end{matrix}\right) = \left(\begin{matrix}
 		\gamma_2&&-\gamma_2\beta_2&&0&&0\\
 		-\gamma_2\beta_2&&\gamma_2&&0&&0\\
 		0&&0&&1&&0\\
 		0&&0&&0&&1
 	\end{matrix}\right)\left(\begin{matrix}
 	x_0'\\
 	x_1'\\
 	x_2'\\
 	x_3'
 \end{matrix}\right)
 \end{equation}
 
   \begin{equation}\nonumber
 	\left(\begin{matrix}
 		x_0''\\
 		x_1''\\
 		x_2''\\
 		x_3''
 	\end{matrix}\right) = \left(\begin{matrix}
 		\gamma_2&&-\gamma_2\beta_2&&0&&0\\
 		-\gamma_2\beta_2&&\gamma_2&&0&&0\\
 		0&&0&&1&&0\\
 		0&&0&&0&&1
 	\end{matrix}\right)\left(\begin{matrix}
 	\gamma_1&&-\gamma_1\beta_1&&0&&0\\
 	-\gamma_1\beta_1&&\gamma_1&&0&&0\\
 	0&&0&&1&&0\\
 	0&&0&&0&&1
 \end{matrix}\right)\left(\begin{matrix}
 x_0\\
 x_1\\
 x_2\\
 x_3
\end{matrix}\right)
 \end{equation}
 
 chamando essa dupla transformação de Lorentz de $\Lambda'$, temos
 
 \begin{equation}\nonumber
 	\Lambda' = \left(\begin{matrix}
 		\gamma_2&&-\gamma_2\beta_2&&0&&0\\
 		-\gamma_2\beta_2&&\gamma_2&&0&&0\\
 		0&&0&&1&&0\\
 		0&&0&&0&&1
 	\end{matrix}\right)\left(\begin{matrix}
 		\gamma_1&&-\gamma_1\beta_1&&0&&0\\
 		-\gamma_1\beta_1&&\gamma_1&&0&&0\\
 		0&&0&&1&&0\\
 		0&&0&&0&&1
 	\end{matrix}\right)
 \end{equation}
calculando a multiplicação de matrizes acima, obtemos


\begin{equation}\nonumber
	\Lambda' = \left(\begin{matrix}
		\gamma_2\gamma_1(1+\beta_1\beta_2)&&-\gamma_2\gamma_1(\beta_1+\beta_2)&&0&&0\\
		-\gamma_2\gamma_1(\beta_1+\beta_2)&&\gamma_2\gamma_1(1+\beta_1\beta_2)&&0&&0\\
		0&&0&&1&&0\\
		0&&0&&0&&1
	\end{matrix}\right)
\end{equation}

calculando explicitamente quem é $ \gamma_2\gamma_1(1+\beta_1\beta_2) $, temos

\begin{eqnarray}
	\gamma_2\gamma_1(1+\beta_1\beta_2) = \left(1 -\frac{v_1^2}{c^2}\right)^{-1/2}\left(1 -\frac{v_2^2}{c^2}\right)^{-1/2}(1+\beta_1\beta_2) = \left(1 -\beta_1^2\right)^{-1/2}\left(1 -\beta_2^2\right)^{-1/2}(1+\beta_1\beta_2) \nonumber 
\end{eqnarray}

$$\gamma_2\gamma_1(1+\beta_1\beta_2) =\left(1 -\beta_1^2\right)^{-1/2}\left(1 -\beta_2^2\right)^{-1/2}(1+\beta_1\beta_2)$$

$$\gamma_2\gamma_1(1+\beta_1\beta_2) =\left(\left(1 -\beta_1^2\right)\left(1 -\beta_2^2\right)\right)^{-1/2}(1+\beta_1\beta_2)$$

$$\gamma_2\gamma_1(1+\beta_1\beta_2) =\left(\left(1 -\beta_1\right)\left(1 +\beta_1\right)\left(1 -\beta_2\right)\left(1 -\beta_2\right)\right)^{-1/2}(1+\beta_1\beta_2)$$

$$\gamma_2\gamma_1(1+\beta_1\beta_2) =\left((1 -(\beta_2\beta_1)^2)^2\right)^{-1/2}(1+\beta_1\beta_2)$$

$$\gamma_2\gamma_1(1+\beta_1\beta_2) =(1 -(\beta_2\beta_1)^2)^{-1}(1+\beta_1\beta_2)= (1 -(\beta_2\beta_1))^{-1}(1 +(\beta_2\beta_1))^{-1}(1+\beta_1\beta_2)$$

$$\gamma_2\gamma_1(1+\beta_1\beta_2) = (1 -(\beta_2\beta_1))^{-1}$$

calculando $-\gamma_2\gamma_1(\beta_1+\beta_2)  $ explicitamente, temos

$$-\gamma_2\gamma_1(\beta_1+\beta_2)=- (1 -(\beta_2\beta_1)^2)^{-1} (\beta_1+\beta_2)$$

A partir daqui não tenho uma ideia clara de como resolver esse exercício.
\end{document}