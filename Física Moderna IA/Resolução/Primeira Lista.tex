\documentclass[10pt,a4paper]{article}
\usepackage[utf8]{inputenc}
\usepackage[T1]{fontenc}
\usepackage{amsmath}
\usepackage{amssymb}
\usepackage{makeidx}
\usepackage{graphicx}
\usepackage[portuguese]{babel}
\author{Arthur de Souza Molina e Gabriel Capelini Magalhaes}
\begin{document}
	\section{Terceiro Exercício}
	A distância até a estrela mais distante da nossa galáxia é da ordem de $10^{5}$ anos-luz. Explique por que é possível, em princípio, para um ser humano viajar para tal estrela durante seu tempo de vida (digamos 80 anos) e faça uma estimativa da velocidade necessária para isso.
	\paragraph{Resposta:}
	Antes de começar qualquer cálculo, perceba que as medidas de intervalo de tempo e de distância são de um observador em repouso no referencial da Terra. Este problema é análogo ao paradoxo dos gêmeos.
	
	Podemos definir o observador na Terra como o referencial S ( o observador na estrela está em repouso em relação ao referencial S) e o observador (o viajante) da nave em repouso em seu respectivo referencial S' que se distância da Terra com a velocidade iremos estimar, isto é, a velocidade relativa entre os referenciais S e S', lembrando que o sentido da velocidade do foguete se altera na mudança de referencial, logo,
	
	\begin{equation}\label{v_foguete}
		  v' = \dfrac{\Delta x'}{\Delta t'}
	\end{equation}
	
	Sabendo que o observador na Terra realiza medidas de comprimento contraídas e intervalos de tempo dilatadas
	\begin{equation}
		\Delta x \approx 10^5 \text{  anos-luz    e  } \Delta t \approx \text{80 anos},
	\end{equation}
	podemos usar as TL para traduzir as medidas para o sistema de coordenadas para S' com
	\begin{equation}\label{TL_repouso}
		\Delta x' = \dfrac{\Delta x}{\gamma}\text{  e  } \Delta t' = \gamma\Delta t.
	\end{equation}

	Ao substituirmos temos
	\begin{eqnarray}
		&& v' = \dfrac{\dfrac{\Delta x}{\gamma}}{\gamma\Delta t} \nonumber \\
		&& v' = \dfrac{\Delta x}{\gamma^2\Delta t}\nonumber
	\end{eqnarray}
	para facilitar as contas, podemos realizar a seguinte operação
	
	\begin{equation}\nonumber
		\dfrac{\Delta x}{\Delta t } = \dfrac{10^5 \text{  anos-luz}}{80\text{ anos}} = \dfrac{10^5c}{80} = 1,25\cdot10^3c
	\end{equation}
	irei definir $ \alpha\equiv 1,25\cdot10^3 $, apenas para não carregar um termo numérico imenso e dar o trabalho de manipula-lo. Retornando as contas temos
	
	\begin{eqnarray}
	    && v' = \dfrac{\alpha c}{\gamma^2}\nonumber \\
		&& v' = \alpha c \left( 1 - \dfrac{v'^2}{c^2}\right) \nonumber \\
		&& v'  =  \alpha c - \dfrac{ \alpha v'^2}{c} \nonumber \\		
		&& \dfrac{ c v'}{\alpha} = c^2 - v'^2 \nonumber \\
		&& v'^2 - \dfrac{ c v'}{\alpha} - c^2 = 0, \nonumber
	\end{eqnarray}
	usando Baskara, temos
	\begin{eqnarray}
		&& v' = -\dfrac{\frac{c}{\alpha} \pm \sqrt{\left(\frac{c}{\alpha}\right)^2 +4c^2}}{2} \nonumber \\
		&& v' = \dfrac{-\frac{c}{\alpha} \pm c\sqrt{\frac{1}{\alpha^2} +4}}{2} \nonumber 
	\end{eqnarray}
	como isso é uma estimativa, podemos considerar que $ \alpha^{-2} \approx 0 $, logo
	\begin{eqnarray}
		&& v' = \dfrac{-\frac{c}{\alpha} \pm c\sqrt{4}}{2} \nonumber \\
		&& v' = c\dfrac{-\frac{1}{\alpha} \pm 2}{2} \nonumber\\
		&& v' = c\left(-\frac{1}{2\alpha} \pm 1\right) \nonumber 
	\end{eqnarray}
	O primeiro resultado, o negativo, temos
	$ v' = c\left(-\frac{1}{2\alpha} - 1\right) \Longrightarrow |v'| > |c|$, ultrapassando a velocidade da luz, na direção contrária da viagem e isso é incorreto.
	
	Já o segundo resultado, o positivo, temos
	$ v' = c\left(-\frac{1}{2\alpha} + 1\right) \approx 0,9996 c$ que é o resultado correto.

\end{document}