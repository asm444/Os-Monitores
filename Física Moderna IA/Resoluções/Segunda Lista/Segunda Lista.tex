\documentclass[10pt,a4paper]{article}
\usepackage[bottom = 2cm, top = 2cm, left = 2cm, right = 2cm]{geometry}
\usepackage[utf8]{inputenc}
\usepackage[T1]{fontenc}
\usepackage{amsmath}
\usepackage{amssymb}
\usepackage{makeidx}
\usepackage{graphicx}
\usepackage{siunitx}
\usepackage{mathtools}
\usepackage[portuguese]{babel}
\author{Arthur de Souza Molina e Gabriel Capelini Magalhaes}
\title{Resolução da Segunda Lista de Exercícios}
\begin{document}
	\maketitle
	
	\begin{enumerate}
	\item Mostre que uma transformação de Lorentz preserva a antissimetria de um tensor, i.e, se $B^{\mu \nu}$ é antissimétrico, então $\tilde{B}^{\mu \nu}$ transformado também será antissimétrico.
		\paragraph{Solução: } Sabendo que $B^{\mu \nu}=-B^{ \nu \mu}$, aplicando uma transformação de Lorentz, temos 
		
		\begin{equation}
		\tilde{B}^{\mu \nu} = \Lambda^{\mu}_{\;\alpha} \Lambda^{\nu}_{\;\beta}B^{\alpha \beta}
		\end{equation}
		Trocando a ordem os índices no lado direito
		
		\begin{equation}
		\tilde{B}^{\mu \nu} = -\Lambda^{\mu}_{\;\beta} \Lambda^{\nu}_{\;\alpha}B^{\beta \alpha}
		\end{equation}
		
Veja que, pelos índices estarem somados, podemos renomealos  de volta como $\alpha \rightarrow \beta$ e $\beta \rightarrow \alpha$, de forma que 

\begin{equation}
		\tilde{B}^{\mu \nu} = -\Lambda^{\mu}_{\;\alpha} \Lambda^{\nu}_{\;\beta}B^{\alpha \beta}
		\end{equation}
		Ou seja, a TL preserva a simetria do tensor $B^{\mu \nu}$. 
	\end{enumerate}
\end{document}