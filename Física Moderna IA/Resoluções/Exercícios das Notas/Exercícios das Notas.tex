\documentclass[10pt,a4paper]{article}
\usepackage[top=2cm, bottom= 2cm, left=2cm, right=2cm]{geometry}
\usepackage[utf8]{inputenc}
\usepackage[T1]{fontenc}
\usepackage{amsmath}
\usepackage{amssymb}
\usepackage{float}
\usepackage{makeidx}
\usepackage{graphicx}
\usepackage{siunitx}
\usepackage{hyperref}
\usepackage{mathtools}
\usepackage[portuguese]{babel}
\usepackage{siunitx}


\newtheorem{exercise}{Exercício}


\hypersetup{
hidelinks = true
}

\title{Resolução dos exercícios das notas - Física Moderna 1 A}

\author{Arthur Souza e Gabriel Capelini}

\begin{document}

\maketitle

\section*{Aula 12}

\paragraph{Exercício 9} Calcule $\eta^{\mu}\eta_{\mu \nu}\eta^{\nu}$ e mostre que é um invariante.

\paragraph{Solução:} 
\begin{equation}\label{norma quadrivelocidade}
\begin{split}
&\eta^{\mu}\eta_{\mu \nu}\eta^{\nu} = \eta^{\mu}\eta_{\nu} = -\eta^0 \eta_0 + \eta^i \eta_i = -\frac{dx^0}{d\tau}\frac{dx_0}{d\tau} + \frac{dx^i}{d\tau}\frac{dx_i}{d\tau} = \gamma^2 \left(-\frac{cdt}{dt}\frac{cdt}{dt} + \frac{dx^i}{dt}\frac{dx_i}{dt}\right) = \gamma^2(-c^2+v^2)\\
& = -\gamma^2 c^2\underbrace{\left(1 - \frac{v^2}{c^2}\right)}_{\gamma^{-2}} = -c^2
\end{split}
\end{equation}
Para mostrar que é um invariante, aplicamos uma TL


\begin{equation}\label{TL arbitrária}
\begin{split}
&\eta^{\mu'}\eta_{\mu \nu}\eta^{\nu'} = \frac{dx^{\mu'}}{d\tau}\eta_{\mu \nu}\frac{dx^{\nu'}}{d\tau} = \frac{d(\Lambda^{\mu}_{\;\;\alpha}x^{\alpha})}{d\tau}\eta_{\mu \nu}\frac{d(\Lambda^{\nu}_{\;\;\beta}x^{\beta})}{d\tau} = \Lambda^{\mu}_{\;\;\alpha}\eta_{\mu \nu}\Lambda^{\nu}_{\;\;\beta}\frac{dx^{\alpha}}{d\tau}\frac{dx^{\beta}}{d\tau} = \eta_{\alpha \beta}\frac{dx^{\alpha}}{d\tau}\frac{dx^{\beta}}{d\tau}\\
& = \eta^{\alpha}\eta_{\alpha \beta}\eta^{\beta}
\end{split} 
\end{equation}
Como a tranformação (\ref{TL arbitrária}) é completamente arbitrária, então $\eta^{\mu}\eta_{\mu \nu}\eta^{\nu}$ é um invariante.


\paragraph{Exercício 10} Explique por que a massa de repouso é um invariante.

\paragraph{Solução:}  Como o resultado do produto interno é um escalar e escalares possuem o mesmo valor visto de qualquer referencial, o resultado $m^2c^2$ é um invariante. Como $c$ é um invariante, então $m$ também deve ser um invariante para essa quantidade como um todo ser invariante. 


\section*{Aula 14}

\paragraph{Exercício 2} Um pion neutro (isso é importante pela conservação de carga) decai em dois fótons. Sabendo que a massa de repouso do pion vale $2,4\times 10^{-28}\unit{kg}$, qual o momento dos fótons criados. É possível que o pion decaia em apenas um
fóton? Justifique sua resposta.

\paragraph{Solução:} Sabemos que o momento relativístico deve se conservar, isso se traduz como

\begin{equation}\label{conservação momento}
 p_{\pi} = p_{\gamma_1} + p_{\gamma_2} = p_{\gamma}
\end{equation}
em que $p_{\pi}$ é o momento do pion e $p_{\gamma}$ o momento total dos fótons. O momento total dos fótons pode ser escrito como 

\begin{equation}\label{momento foton}
E_{\gamma} = cp_{\gamma} \Rightarrow p_{\gamma} = \frac{E_{\gamma}}{c}, \quad p_{\gamma} = |\mathbf{p_{\gamma}}|
\end{equation}
Sabendo que a seguinte quantidade 

\begin{equation}\label{energia momento}
E^2 - (cp)^2 = (mc^2)^2
\end{equation}
é invariante. Podemos tomar o referencial de repouso do pion (i.e $p =0$), de forma que 

\begin{equation}\label{energia repouso}
E_{\pi} = mc^2
\end{equation}
Então, de (\ref{conservação momento}) segue que

\begin{equation}
p_{\gamma} = \frac{E_{\pi}}{c} = \frac{mc^2}{c} = mc
\end{equation}
Substituindo os valores dados 

\begin{equation}
p_{\gamma} = (2,4\times 10^{-28}\si{kg})(3\times 10^8 \unit{m/s}) = 7,2 \times 10^{-20}\si{kg.m/s}
\end{equation}
Se o pion decair em apenas um fóton, a conservação de momento não é satisfeita, pois o fóton viaja a velocidade da luz e o pion, por ter massa, não é capaz de viajar a velocidade da luz. Dessa forma, outro fóton com direção contrária precisa aparecer para que o momento total seja conservado.

\end{document}