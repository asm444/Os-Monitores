\documentclass[10pt,a4paper]{article}
\usepackage[top=2cm, bottom= 2cm, left=2cm, right=2cm]{geometry}
\usepackage[utf8]{inputenc}
\usepackage[T1]{fontenc}
\usepackage{amsmath}
\usepackage{amssymb}
\usepackage{float}
\usepackage{makeidx}
\usepackage{graphicx}
\usepackage{siunitx}
\usepackage{hyperref}
\usepackage{mathtools}
\usepackage[portuguese]{babel}


\newtheorem{exercise}{Exercício}


\hypersetup{
hidelinks = true
}

\title{Resolução dos exercícios das notas - Física Moderna 1 A}

\author{Arthur Souza e Gabriel Capelini}

\begin{document}

\maketitle

\section*{Aula 12}

\textbf{Exercício 9} Calcule $\eta^{\mu}\eta_{\mu \nu}\eta^{\nu}$ e mostre que é um invariante.

\paragraph{Solução:} 
\begin{equation}\label{norma quadrivelocidade}
\begin{split}
&\eta^{\mu}\eta_{\mu \nu}\eta^{\nu} = \eta^{\mu}\eta_{\nu} = -\eta^0 \eta_0 + \eta^i \eta_i = -\frac{dx^0}{d\tau}\frac{dx_0}{d\tau} + \frac{dx^i}{d\tau}\frac{dx_i}{d\tau} = \gamma^2 \left(-\frac{cdt}{dt}\frac{cdt}{dt} + \frac{dx^i}{dt}\frac{dx_i}{dt}\right) = \gamma^2(-c^2+v^2)\\
& = -\gamma^2 c^2\underbrace{\left(1 - \frac{v^2}{c^2}\right)}_{\gamma^{-2}} = -c^2
\end{split}
\end{equation}

Para mostrar que é um invariante, aplicamos uma TL


\begin{equation}\label{TL arbitrária}
\begin{split}
&\eta^{\mu'}\eta_{\mu \nu}\eta^{\nu'} = \frac{dx^{\mu'}}{d\tau}\eta_{\mu \nu}\frac{dx^{\nu'}}{d\tau} = \frac{d(\Lambda^{\mu}_{\;\;\alpha}x^{\alpha})}{d\tau}\eta_{\mu \nu}\frac{d(\Lambda^{\nu}_{\;\;\beta}x^{\beta})}{d\tau} = \Lambda^{\mu}_{\;\;\alpha}\eta_{\mu \nu}\Lambda^{\nu}_{\;\;\beta}\frac{dx^{\alpha}}{d\tau}\frac{dx^{\beta}}{d\tau} = \eta_{\alpha \beta}\frac{dx^{\alpha}}{d\tau}\frac{dx^{\beta}}{d\tau}\\
& = \eta^{\alpha}\eta_{\alpha \beta}\eta^{\beta}
\end{split} 
\end{equation}
Como a tranformação (\ref{TL arbitrária}) é completamente arbitrária, então $\eta^{\mu}\eta_{\mu \nu}\eta^{\nu}$ é um invariante.


\noindent \textbf{Exercício 10} Explique por que a massa de repouso é um invariante.

\paragraph{Solução:}  Como o resultado do produto interno é um escalar e escalares possuem o mesmo valor visto de qualquer referencial, o resultado $m^2c^2$ é um invariante. Como $c$ é um invariante, então $m$ também deve ser um invariante para essa quantidade como um todo ser invariante. 


\end{document}